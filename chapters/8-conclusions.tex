\glsresetall
\chapter{Conclusion and Future Work}
\label{chap:conclusion_and_future_work}

% FROM PAPER
%This Section covers three main topics. It contains a summary about what was done and what were the main conclusions extracted from this research followed by brief reflections regarding this whole research topic and ending with the future work and research path that we hope to be taken in the future.

%This Chapter contains a summary about what was done, the main conclusions and contributions, followed by a brief reflection about the achievements of this research and the future work that can follow from these achievements.

This Chapter covers three main topics: a summary of what we did and the main conclusions we reached from this research; followed by brief reflections regarding this whole research topic; and ending with the future work and research paths that seem to be promising for the future.

%\subsection{Summary}
%\label{subsec:summary}

%After this whole research, we are able to state that tracing data is useful and required to find anomalies in large-scale distributed systems. However, tracing data is hard to handle due to complexity and information plethora. We have used this information to detect anomalies presented in services. In order do it, we have extracted metrics, based in time, from tracing data and then performed outlier detection analysis over these metrics. Our perception is that, issues addressed in this paper can only be identified using tracing data, nevertheless, it is very expensive to analyse it directly.

After this whole research, we are able to state that tracing data is useful and required to find anomalies related to service morphology. However, this type of data is hard to handle and one must use it if some issue was detected in metrics easier to analyse, e.g. monitoring. For this type of data to be easier to analyse, a discussion is provided about this difficulty bellow. So, in the end our perception is that, there are issues that we can only perceive using tracing data, but it is very expensive to analyse this data directly.

From tracing quality analysis, both tests are very interesting but, due to lack of required and strict specification, the tests and results of the ``structural quality analysis'' using spans are not very useful however, one can state that this is all we can do taking into consideration the \emph{OpenTracing} specification.

%\subsection{Brief reflections}
%\label{subsec:brief_reflections}

In the end, our analysis of the provided tracing data generated by \emph{OpenStack} -- Huawei Cluster, took us to the following conclusions about \emph{OpenTracing}:

\begin{enumerate}
    \item \emph{OpenTracing} suffers from a lack of tools for data processing and visualisation.
    \item The \emph{OpenTracing} specification is ambiguous.
    \item The lack of tools to control instrumentation quality jeopardizes the tracing effort.
\end{enumerate}

%One point to consider is that it was difficult to find tools for tracing data visualisation and processing. Only Zipkin and Jaeger, presented in the Subsection~\ref{subsec:distributed_tracing_tools}, represent some usefulness has they allow distributed tracing visualisation in a more human readable way. However, they do not present any kind of tracing analysis. So, because of this, there are real needs for tools that can handle this kind of data.

Firstly, we found it difficult to find appropriate tools for tracing data processing and visualisation. Only \emph{Zipkin} and \emph{Jaegger}, presented in the Subsection~\ref{subsec:distributed_tracing_tools}, are useful, as they allow distributed tracing visualisation in a human readable way. Unfortunately, they do not present any kind of tracing analysis. The need for additional open-source tools that can perform tracing analysis and visualisation is therefore quite real.

%The second point is the big problem with the OpenTracing specification. The main difficulties in implementing the tools mentioned in this paper were felt because of the ambiguity in tracing data. The specification allows many fields that are not strictly defined. As mentioned in Section~\ref{sec:second_question}, one of the detected problems are in the measurement units. Other problem resides in some fields that contain very important information about the path of the request, these fields are defined as maps of key - values, where the keys can be anything that the programmer wants. This brings a big problem, because tools must perceive this kind of values and some of them may not be considered for analysis. A simple solution could be to redefine the specification and reduce these kind of fields, transforming the specification into a more strict schema. This would allow the implementation of more general trace processing tools.

Secondly, one of the main difficulties in implementing the \gls{otp} and Data Analysis tools we mentioned in this thesis is the ambiguity in tracing data. The specification includes many fields that are not strictly defined. As mentioned in Section~\ref{sec:trace_quality_analysis}, one of the problems is the lack of standardization of measurement units, which led to different ones being used in the data provided. Other problem resides in some fields that contain very important information about the path of the request. These fields are defined as key-value pairs, where the keys vary freely according to the programmer's needs. This raises a major challenge for tools, which must infer the units, or assume that some data is unsuitable for analysis. A simple solution could be to redefine the specification and reduce this kind of fields, transforming the specification into a more strict schema. This would allow the implementation of more general trace processing tools.

%\subsection{Future work}
%\label{subsec:future_work}

Therefore, from this work, the following research paths are considered for future work:

\begin{enumerate}
    \item Improve and develop new tools for \emph{OpenTracing} processing.
    \item Perform a research to redefine the \emph{OpenTracing} specification.
    \item Explore and analyse the remaining extracted tracing metrics.
    \item Use tracing data from other systems.
    \item Develop a simulated system with the capability of fault-injection to prove the analysis observations.
    \item Conciliate the results from tracing data with other kinds of data like monitoring and logging.
    \item Follow closely the development and the community of \emph{OpenTelemetry} project, and contribute with ideas generated by this research.
\end{enumerate}

First, today there are not many tools for processing and handling \emph{OpenTracing} data. This increased difficulty is felt when we needed to process this kind of data in a different way, because we always ended up developing everything from scratch.

Second, there must be a way to eliminate or reduce the ambiguity and uncertainty of data presented in tracing generated by non-strict fields. If the specification can not be changed, a new way to transform tracing data to ease the analysis is very welcome. However, this is a topic that should be covered by the development of \emph{OpenTelemetry} project, as mentioned in Section~\ref{sec:limitations_of_opentracing_data}.

Third, these developed tools extract many more metrics. The majority of them were not explored due to lack of time, and therefore, here resides the opportunity to do it. The path starts by defining new research questions or analyse the remaining ones, presented in Section~\ref{sec:research_questions}, that use these metrics and develop ways to analyse them.

Fourth, just one data set of tracing data was used in this research. Test the tools and methods with other tracing data could be an interesting path.

Fifth, the system were the data was gathered was a company testing system. One good future approach was to have a microservice based simulated system, were the developers could inject faults like request flow redirection, latency issues, and others, point them out and test the developed tools and methods.

Sixth, only tracing data was used in this research, one interesting path to follow is to have more kinds of data like monitoring and logging from the target system. This could help the analysis of the system, due to more knowing about it.

Seventh, after developed, \emph{OpenTelemetry} solution could cover points 2 and 5 mentioned here. Also, expectation of success is high in the community, and commitment is visible in the project pulse. For these reasons, this project is a must watch in the following months.

%\newpage

We started with only tracing data provided by Huawei, and walked a path were we defined research questions based in \gls{devops} needs and in \emph{OpenTracing} characteristics. Later on, we designed a proposed solution capable of processing tracing data and extract metrics from this type of data. Then we implemented this solution and used it to retrieve results. These results proved some issues presented in \emph{OpenTracing} specification and the difficulty that is to analyse a distributed system only using tracing information.

\emph{OpenTelemetry} was created and started near the end of the research work presented in this thesis, with the core objective of merging \emph{OpenCensus} and \emph{OpenTracing} into a single \gls{api}, and consequently, review both specifications in order to modify and improve them. This projects emphasizes the work performed in this research, because the raised problems in this thesis are covered by it.

In the end, given the imposed limitations, one may conclude that this work was a success because the research directions are in the vanguard of the state of the art, related work and general community of tracing usage and analysis. Also, tools for tracing processing and analysis were developed and the created methods and conclusions were used to produce a scientific paper submitted to the International Symposium on Network Computing and Applications (IEEE NCA 2019).

%\section{Brief Reflection}
%\label{sec:brief_reflection}
%
%\todo{This thesis presents an attempt to further the field of distributed system %based tracing analysis.}
%
%\todo{...}
%
%\todo{A reflection about the tools and methods produced and the open paths from %this whole research are exposed. Also a reflection of the main difficulties %felted with this research are presented}
%
%\section{Future Work}
%\label{sec:future_work}
%
%\todo{Talk a bit about the possible work in this field and what can be achieved %using this work.}
%
%\todo{the future work that can be addressed considering this work is properly %explained taking into consideration what is said in the previous section}
%
%\section{Concluding Research Questions}
%\label{sec:concluding_research_questions}
%
%\todo{...}
%
%\todo{EXTRACTED FROM PAPER! (KEEP ONLY THIS?)}

\checkoddpage
\ifthenelse{\boolean{oddpage}}
{
    \newpage
    \blankpage
}
{
    % Even page
}