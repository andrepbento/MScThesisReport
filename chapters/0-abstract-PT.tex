% --------------- Notes ----------------------
% - Max. 300 palavras;
% - 5(CINCO) Palavras-chave;
% --------------------------------------------
\newgeometry{left=8cm}

\section*{Resumo}
\label{sec:resumo}

% [Old Abstract] Hoje em dia encontramos-nos num mundo em que a crescente evolução tecnológica exige cada vez mais dos sistemas computacionais e especialmente das pessoas que os desenvolvem e mantêm. Com este crescimento, certas características como a complexidade e a distribuição dos sistemas aumentam a passos largos, de modo a que se torna bastante difícil de os gerir e de perceber o seu funcionamento em geral. É neste problema que este trabalho visa prestar soluções. As soluções apresentadas neste documento, têm como principal objetivo melhorar a forma como os administradores deste género de sistemas, encaram e resolvem os problemas que estão a ocorrer nos seus sistemas. Para gerar estas soluções foi desenvolvido um trabalho de pesquisa, onde foram estudados sistemas e metodologias existentes que se adequam ao tratamento da informação gerada por sistemas baseados em microserviços. Por fim é implementado um sistema que integra as soluções apresentadas, com o objectivo de demonstrar a aplicabilidade prática como prova de conceito.

% 1. one sentence presenting the topic (make readers familiar)
A arquitetura de software baseada em micro-serviços está a crescer em uso e um dos tipos de dados gerados para manter o histórico do trabalho executado por este tipo de sistemas é denominado de tracing.
% 2. one sentence with problem statement and key research question
Mergulhar nestes dados é díficil devido à sua complexidade, abundância e falta de ferramentas. Consequentemente, é díficil para os \gls{devops} de analisarem o comportamento dos sistemas e encontrar serviços defeituosos usando tracing.
% 3. one sentence justifying why the research question is challenging, valid, and unanswered
Hoje em dia, as ferramentas mais gerais e comuns que existem para processar este tipo de dados, visam apenas apresentar a informação de uma forma mais clara, aliviando assim o esforço dos \gls{devops} ao pesquisar por problemas existentes nos sistemas, no entanto estas ferramentas não fornecem bons filtros para este tipo de dados, nem formas de executar análises dos dados e, assim sendo, não automatizam o processo de procura por problemas presentes no sistema, o que gera um grande problema porque recaem nos utilizadores para o fazer manualmente.
% 4. one sentence explaining our approach to solve the problem
Nesta tese é apresentada uma possivel solução para este problema, capaz de utilizar dados de tracing para extrair metricas do grafo de dependências dos serviços, nomeadamente o número de chamadas de entrada e saída em cada serviço e os tempos de resposta coorepondentes, com o propósito de detectar qualquer serviço defeituoso presente no sistema e identificar as falhas em espaços temporais especificos. Além disto, é apresentada também uma possivel solução para uma análise da qualidade do tracing com foco em verificar a qualidade da estrutura do tracing face à especificação do OpenTracing e a cobertura do tracing a nível temporal para serviços especificos.
% 5. one sentence explaining the method (experiments, prototypes, models, ...)
A abordagem que seguimos para resolver o problema apresentado foi implementar ferramentas protótipo para processar dados de tracing de modo a executar experiências com as métricas extraidas do tracing fornecido pela Huawei.
% 6. one sentence outlining expected impact or implications for practice
Com esta proposta de solução, esperamos que soluções para processar e analisar tracing comecem a surgir e a serem integradas em ferramentas de sistemas distribuidos.

\section*{Palavras-Chave}
\label{sec:palavras}

Micro-serviços, Computação na nuvem, Observabilidade, Monitorização, Tracing.

\restoregeometry