% --------------- Notes ----------------------
% - Max. 300 words;
% - 5(FIVE) Key-words;
% --------------------------------------------
\newgeometry{left=8cm}

\section*{Abstract}
\label{sec:abstract}

% [Old Abstract] Nowadays, we find ourselves in a world in which the increasing technological evolution demands more of the computational systems and specially, of the people who develop and maintain them. With this growth, certain characteristics such as the size and distribution of the systems increase, therefore, it becomes quite difficult to manually manage them and to inspect problems in their operation due to complexity and increase of data produced. It is in this problem field that this research work aims to provide solutions. One type of data generated by distributed systems is called tracing data. The main objective of the solutions presented in this document are to relieve from the user the task of finding system anomalies using tracing data, and to perform an inspection about the quality of tracing. In order to generate these solutions, a research work was developed to study existing systems and methodologies that are adequate to the treatment of the information generated by systems based on microservices. Finally, a system that integrates the presented solutions is implemented, with the objective of demonstrating the practical application as proof of concept.

% 1. one sentence presenting the topic (make readers familiar)
Microservice based software architecture are growing in usage and one type of data generated to keep history of the work performed by this kind of systems is called tracing data. Tracing can be used to help \gls{devops} perceive problems such as latency and request work-flow in their systems.
% 2. one sentence with problem statement and key research question
Diving into this data is difficult due to its complexity, plethora of information and lack of tools. Hence, it gets hard for \gls{devops} to analyse the system behaviour in order to find faulty services using tracing data.
% 3. one sentence justifying why the research question is challenging, valid, and unanswered
The most common and general tools existing nowadays for this kind of data, are aiming only for a more human-readable data visualisation to relieve the effort of the \gls{devops} when searching for issues in their systems. However, these tools do not provide good ways to filter this kind of data neither perform any kind of tracing data analysis and therefore, they do not automate the task of searching for any issue presented in the system, which stands for a big problem because they rely in the system administrators to do it manually.
% 4. one sentence explaining our approach to solve the problem
In this thesis is present a possible solution for this problem, capable of use tracing data to extract metrics of the services dependency graph, namely the number of incoming and outgoing calls in each service and their corresponding average response time, with the purpose of detecting any faulty service presented in the system and identifying them in a specific time-frame. Also, a possible solution for quality tracing analysis is covered checking for quality of tracing structure against OpenTracing specification and checking time coverage of tracing for specific services.
% 5. one sentence explaining the method (experiments, prototypes, models, ...)
Regarding the approach to solve the presented problem, we have relied in the implementation of some prototype tools to process tracing data and performed experiments using the metrics extracted from tracing data provided by Huawei.
% 6. one sentence outlining expected impact or implications for practice
With this proposed solution, we expect that solutions for tracing data analysis start to appear and be integrated in tools that exist nowadays for distributed tracing systems.

\section*{Keywords}
\label{sec:keywords}

Microservices, Cloud Computing, Observability, Monitoring, Tracing.

\restoregeometry