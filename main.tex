\documentclass[twoside, 11pt]{report}

\usepackage[table,xcdraw]{xcolor}
\usepackage[utf8]{inputenc}
\usepackage[english]{babel}
\usepackage[super]{nth}
\usepackage{vhistory}
\usepackage{url}
\usepackage{rotating}
\usepackage{pdflscape}
\usepackage[hidelinks, bookmarks=true]{hyperref}
\usepackage{graphicx}
\graphicspath{{images/}}
\usepackage[acronym]{glossaries}
%\usepackage[acronym,]{glossaries}
%\usepackage{glossaries}
\usepackage{float}
\usepackage{pgfplotstable}
\usepackage{tabularx}
\usepackage{flafter}
% For labeling references
\usepackage{nameref}
% For even/odd pages
\usepackage{ifthen, changepage}
% PDF metadata
\usepackage{hyperref}
\hypersetup{
    pdftitle={Observing and Controlling Performance in Microservices},
    pdfsubject={Master Thesis Report},
    pdfauthor={andrepbento}
}

% Code listing
\usepackage{listings}
\usepackage{color}
 
\definecolor{codegreen}{rgb}{0,0.6,0}
\definecolor{codegray}{rgb}{0.5,0.5,0.5}
\definecolor{codepurple}{rgb}{0.58,0,0.82}
\definecolor{backcolour}{rgb}{0.95,0.95,0.92}

\lstdefinestyle{mystyle}{
    backgroundcolor=\color{backcolour},   
    commentstyle=\color{codegreen},
    keywordstyle=\color{magenta},
    numberstyle=\tiny\color{codegray},
    stringstyle=\color{codepurple},
    basicstyle=\footnotesize,
    breakatwhitespace=false,         
    breaklines=true,                 
    captionpos=b,                    
    keepspaces=true,                 
    numbers=left,                    
    numbersep=5pt,                  
    showspaces=false,                
    showstringspaces=false,
    showtabs=false,                  
    tabsize=2
}
 
\lstset{style=mystyle}

% Algorithm
\usepackage[ruled,vlined,linesnumbered,resetcount]{algorithm2e}
\usepackage{enumitem}

% Colours
%\usepackage[disable]{todonotes}
\definecolor{dkgreen}{rgb}{0,0.6,0}
\definecolor{gray}{rgb}{0.5,0.5,0.5}
\definecolor{mauve}{rgb}{0.58,0,0.82}
\definecolor{gray}{rgb}{0.7,0.7,0.7}
\definecolor{lightgray}{rgb}{0.95,0.95,0.95}
\definecolor{lightcyan}{rgb}{0.6784,0.8741,0.902}

% Table definitions
\usepackage{longtable}
\usepackage[table]{xcolor}
\usepackage{array, booktabs, caption}
\usepackage{cellspace}
\usepackage{makecell}
%\rowcolors{1}{white}{lightgray}
%\renewcommand\theadfont{\bfseries}
\newcolumntype{s}{>{\hsize=.25\hsize}X}

%Don't show appendices
%\usepackage[]{appendix} 
%Don't show appendices on ToC
\usepackage[page]{appendix} 
%Show appendices on both text and ToC
%\usepackage[toc,page]{appendix} 

\usepackage{subcaption}
%margins
\usepackage[a4paper,top=2.5cm,bottom=2.5cm,left=3.5cm,right=2.5cm,headsep=12pt]{geometry}

%Indent first paragraph
%\usepackage{indentfirst}
\usepackage{indentfirst}
%Remove all paragraph indentation
%\setlength\parindent{0pt}

%%%%%%%%%%%%%%%%%%%%%%%%%%%%
\usepackage{silence}
\WarningFilter{biblatex}{File 'english-ieee.lbx'}
\usepackage{csquotes}
\usepackage[citestyle=ieee, bibstyle=ieee, block=none, indexing=false, citereset=none, isbn=true, url=true, doi=true, natbib=true, sorting=none, backend=biber]{biblatex}
\addbibresource{library.bib}

%used for cover page
\usepackage{color}
\definecolor{blueFCTUC}{RGB}{0, 150, 215}

%Right align section titles and add a dot to the section number
%\usepackage{sectsty}
%\sectionfont{\raggedright}
%\usepackage{titlesec}
%\titlelabel{\thetitle.\quad}
%\titlespacing\section{0pt}{0.5cm}{1.5cm}

% Useful commands
% ToDo configuration
\newcommand\todo[1]{\textcolor{red}{#1}}

% \HRule creates an horizontal line with <linewidth> length and 0.2mm thickness, that receives an optional parameter specifying the line's distance from the text
\newcommand{\HRule}[1][\medskipamount]{\par
  \vspace*{\dimexpr-\parskip-\baselineskip+#1}
  \noindent\rule{\linewidth}{0.2mm}\par
  \vspace*{\dimexpr-\parskip-.5\baselineskip+#1+5pt}}
%%%

%\setitemize command will set the spacing between items in itemizes to the given parameter.  
%\let\OLDitemize\itemize
%\newcommand{\setitemize}[1]{
%\renewcommand\itemize{\OLDitemize\setlength{\itemsep}{#1}}
%}
%%%

% Adds a blank page – used to guarantee that chapter always start at an even page.
\newcommand*{\blankpage}{%
\newpage
\thispagestyle{empty}
\vspace*{\fill}
\begin{center} This page is intentionally left blank.
\end{center}
\vspace{\fill}
\newpage}
%%%
% Blank page
\newcommand*{\emptypage}{%
\newpage
\thispagestyle{empty}
\vspace*{\fill}
\vspace{\fill}
\newpage}
%%%

% --------------- headers and footers control ----------------------
\usepackage{fancyhdr}
\pagestyle{fancy}
\renewcommand{\chaptermark}[1]{\markboth{\MakeUppercase{#1}}{}}
 
\fancyhead[LE]{\sl{\nouppercase{\chaptername~\thechapter}}}
\fancyhead[RE]{}
\fancyhead[LO]{}
\fancyhead[RO]{\sl{\nouppercase{\leftmark}}}

%\fancyhead[RE,LO]{\rightmark}
\renewcommand{\headrulewidth}{0.4pt}

% Glossaries and Acronyms
\newacronym{api}{API}{Application Programming Interface}
\newacronym{cpu}{CPU}{Central Processing Unit}
\newacronym{dei}{DEI}{Department of Informatics Engineering}
\newacronym{devops}{DevOps}{Development and Operations}
\newacronym{gdb}{GDB}{Graph Database}
\newacronym{http}{HTTP}{Hypertext Transfer Protocol}
\newacronym{otp}{OTP}{OpenTracing processor}
\newacronym{qa}{QA}{Quality Attribute}
\newacronym{rpc}{RPC}{Remote Procedure Call}
\newacronym{tsdb}{TSDB}{Time Series Database}
\makenoidxglossaries

% Document begin
\begin{document}

\pagenumbering{gobble}

\begin{titlepage}
	\centering
	{\scshape\LARGE Master’s Degree in Informatics Engineering \par}
	
	\vspace{0.2cm}
	{\scshape Final Dissertation\par}
	
	\vspace{1.5cm}
	{\huge\bfseries Observing and Controlling Performance in Microservices\par}
	
	\vspace{2cm}
	{Author:\par \Large\itshape André Pascoal Bento\par}
	
	\vfill
	Supervisor:\par
	\large Prof. Filipe João Boavida Mendonça Machado Araújo\par
	
	\vspace{0.5cm}
	Co-Supervisor:\par
	\large Prof. António Jorge Cardoso\par
    
    \includegraphics[width=0.5\textwidth]{fctuc-logo.pdf}\par
    
    \vspace{0.5cm}
	\vfill

	{\Large July 2019\par}
\end{titlepage}
\blankpage

%Gobble Cover, TOC and acknowledgement section pages
\pagenumbering{gobble}

%\blankpage

\pagenumbering{roman}

%Set paragraph spacing
\setlength{\parskip}{0.7em}

%Abstract
% --------------- Notes ----------------------
% - Max. 300 words;
% - 5(FIVE) Key-words;
% --------------------------------------------
\newgeometry{left=8cm}

\section*{Abstract}
\label{sec:abstract}

Nowadays, we find ourselves in a world in which the increasing technological evolution demands more and more of the computational systems and specially, of the people who develop and maintain them. With this growth, certain characteristics such as the complexity and the distribution of the systems increase in stride, so that it becomes quite difficult to manage them and to perceive their operation in general. It is in this problem that this work aims to provide solutions. The main objective of the solutions presented in this document are to improve the way in which the administrators of this kind of systems, face and solve the problems that are occurring in their systems. In order to generate these solutions, a research work was developed to study existing systems and methodologies that are adequate to the treatment of the information generated by systems based on microservices. Finally, a system that integrates the presented solutions is implemented, with the objective of demonstrating the practical application as proof of concept.

\section*{Keywords}
\label{sec:keywords}

Microservices, Cloud Computing, Observability, Monitoring, Tracing.

\restoregeometry
\blankpage
% --------------- Notes ----------------------
% - Max. 300 palavras;
% - 5(CINCO) Palavras-chave;
% --------------------------------------------
\newgeometry{left=8cm}

\section*{Resumo}
\label{sec:resumo}

Hoje em dia encontramos-nos num mundo em que a crescente evolução tecnológica exige cada vez mais dos sistemas computacionais e especialmente das pessoas que os desenvolvem e mantêm. Com este crescimento, certas características como a complexidade e a distribuição dos sistemas aumentam a passos largos, de modo a que se torna bastante difícil de os gerir e de perceber o seu funcionamento em geral. É neste problema que este trabalho visa prestar soluções. As soluções apresentadas neste documento, têm como principal objetivo melhorar a forma como os administradores deste género de sistemas, encaram e resolvem os problemas que estão a ocorrer nos seus sistemas. Para gerar estas soluções foi desenvolvido um trabalho de pesquisa, onde foram estudados sistemas e metodologias existentes que se adequam ao tratamento da informação gerada por sistemas baseados em microserviços. Por fim é implementado um sistema que integra as soluções apresentadas, com o objectivo de demonstrar a aplicabilidade prática como prova de conceito.

\section*{Palavras-Chave}
\label{sec:palavras}

Micro-serviços, Computação na nuvem, Observabilidade, Monitorização, Tracing.

\restoregeometry
\blankpage

%Acknowledgements
\newgeometry{left=5cm}

\section*{Acknowledgements}
\label{sec:acknowledgements}


This work wouldn't be possible to be accomplished without the help and support of a lot of people. Thus, in this section I'd like to give my sincere thanks to all of them.

Starting by giving my thanks to my mother and to my whole family who have supported me through my entire academic course, and who always gave and always will give me some of the most important and beautiful things in life, love and friendship.

In second place I would like to thank all the people that were involved directly to this project. To my supervisor, Professor Filipe Araújo, who contributed with his vast wisdom and experience, to my co-supervisor, Professor Jorge Cardoso, who contributed with is vision and guidance about the main road we should take and to Engineer Jaime Correia, who ``breathes'' these kind of topics through him and helped a lot with is enormous knowledge and enthusiasm.

In third place, I would like to thank the Department of Informatics Engineering and the Centre for Informatics and Systems, both from the University of Coimbra, for allowing and provide the resources and facilities for this project to be carried out.

In fourth place, to the Foundation for Science and Technology, for financing this project facilitating its accomplishment and to Huawei, for providing data, core for this whole research.

And finally, my sincere thanks to everyone that I have not mentioned and contributed to everything that I am today.


\restoregeometry
\blankpage

% Set paragraph spacing to use in ToC and other lists
\setlength{\parskip}{0em}

% Make sure acronym don't appear in the ToC
\glsunsetall
\tableofcontents
\glsresetall

%\newpage
%\blankpage

% List of acronyms
\glsnogroupskiptrue
\printnoidxglossaries
\newpage

\blankpage

\listoffigures

\blankpage
\listoftables

\blankpage

%Begin page numbering
\pagenumbering{arabic}

% Set paragraph spacing
\setlength{\parskip}{0.7em}

\glsresetall
\chapter{Introduction}
\label{chap:introduction}

This document presents the \textit{Master Thesis} in \textit{Informatics Engineering} of the student \textit{André Pascoal Bento} during the school year of 2018/2019, taking place in the \textit{\gls{dei}} of the \textit{University of Coimbra}.

\section{Context}
\label{sec:context}

In today's world, software systems tend to become more distributed as time move on, resulting in new approaches that lead to new solutions and new patterns of developing software. One way to solve this is to develop systems that have their components decoupled, creating software with ``small pieces'' connected to each other that encapsulate and provide a specific function in the larger service. This way of developing software is called Microservices and has become mainstream in the enterprise software development industry~\cite{Dragoni2017}. However, with this kind of approach, the systems complexity is increased as a whole because with more ``small pieces'', more connections are needed and with this more problems related to latency and requests become harder to detect, analyse and correct~\cite{DiFrancesco2017}.

To keep a history of the work performed by this kind of systems, multiple techniques like monitoring~\cite{Joyce1987}, logging~\cite{logging} and tracing~\cite{distributed_tracing} are adopted. Monitoring consists on measuring some aspects like, e.g., \gls{cpu} usage, hard drive usage and network latency of the entire system or of some specific node in a distributed system. Logging provides an overview to a discrete, event-triggered log. Finally, tracing is much similar to logging, however the focus is  registering the flow of execution of the program through several system modules and boundaries. Lastly, distributed tracing, shares the focus on preserving causality relationships, however, is geared towards the modern distributed environments, where state is partitioned over multiple, threads, processes, machines and even geographical locations. This last one is better explained in Subsection~\ref{subsec:distributed_tracing} - \nameref{subsec:distributed_tracing}. There are multiple approaches to gather information of this kind of systems, each with its benefits and disadvantages.

The main problem with this nowadays is that there are not many implemented tools for processing tracing data and none for performing analysis of this type of data. For monitoring it tend to be easier, because data is represented in charts and diagrams, however for logging and tracing it gets harder to manually analyse the data due to multiple factors like its complexity, plethora and increasing quantity of information. There are some visualisation tools for the \gls{devops} to use, like the ones presented in Subsection~\ref{subsec:distributed_tracing_tools}~-~\nameref{subsec:distributed_tracing_tools}, however none of them gets to the point of performing the analysis of the system using tracing, has they tend to be developed only for visualisation and display of tracing data in a more human readable way. Nevertheless, this is critical information about the system behaviour, and thus there is the need for performing automatic tracing analysis.

\section{Motivation}
\label{sec:motivation}

The motivation behind this work resides in exploring and develop ways to perform tracing analysis in microservice based systems. The analysis of this kind of systems tend to be very complex and hard to perform due to their properties and characteristics, as it is explained in Subsection~\ref{subsec:microservices} - \nameref{subsec:microservices}, and to the type of data to be analysed, presented in Subsections~\ref{subsec:distributed_tracing} - \nameref{subsec:distributed_tracing} and~\ref{subsec:traces_and_spans} - \nameref{subsec:traces_and_spans}.

\gls{devops} teams have lots of problems when they need to identify and understand problems with this systems. They usually detect the problems when the client complains about the quality of service, and after that \gls{devops} dive in monitoring metrics like, e.g, \gls{cpu} usage, usage, hard drive usage and network latency, and then in distributed tracing data visualisations and logs to find some explanation to what is causing the reported problem. This involves a very hard and tedious work of look-up through lots of data that represents the history of work performed by the system and, in most cases, this tedious work reveals like a big ``find a needle in the haystack'' problem. Some times, \gls{devops} can only perceive problems in some services and end up ``killing'' and rebooting these services which is wrong, however, due to lack of time and difficulty in identifying anomalous services precisely this is the best known approach.

Problems regarding the system operation are more common in distributed systems and their identification must be simplified. This need of simplification comes from the exponential increase in the amount of data needed to retain information and the increasing difficulty in manually managing distributed infrastructures. The work presented in this thesis, aims to perform a research around these needs and focus on presenting some solutions and methods to perform tracing analysis.

\section{Goals}
\label{sec:goals}

The main goals for this thesis consists on the main points exposed bellow:

\begin{enumerate}
    \item Search for existing technology and methodologies used to help \gls{devops} teams in their current daily work, with the objective of gathering the best practices about handling tracing data. Also, we aim to understand how these systems are used, what are their advantages and disadvantages to better know how we can use them to design and produce a possible solution capable of performing tracing analysis. From this we expect to learn the state of the field for this research, covering the core concepts related work and technologies, presented in Chapter~\ref{chap:state_of_the_art}~-~\nameref{chap:state_of_the_art}.
    \item Perform a research about the main needs of \gls{devops} teams, to better understand what are their biggest concerns that lead to their approaches when performing pinpointing of microservices based systems problems. Relate these approaches with related work in the area, with the objective of understanding what other companies and groups have done in the field of automatic tracing analysis. The processes used to tackle this type of data, their main difficulties and conclusions provide a better insight about the problem. From this we expected to have our research objectives clearly defined and a compilation of questions to be evaluated and answered, presented in Chapter~\ref{chap:research_objectives_and_approach}~-~\nameref{chap:research_objectives_and_approach}.
    \item Reason about all the gathered information, design and produce a possible solution that provides a different approach to perform tracing analysis. From this we expect first to propose a possible solution, presented in Chapter~\ref{chap:possible_solution}. The we implement it using state of the art technologies, feed it with tracing data provided by Huawei and collect results, presented in Chapter~\ref{chap:implementation_process}~-~\nameref{chap:implementation_process}. Finally, we provide conclusions to this research work in the last Chapter~\ref{chap:conclusions}~-~\nameref{chap:conclusions}.
\end{enumerate}

\section{Research Contributions}
\label{sec:research_contributions}

From the work presented on this thesis, the following research contributions were made:

\begin{itemize}
    \item \todo{Andre Bento, Jaime Correia, Ricardo Filipe, Filipe Araujo and Jorge Cardoso. On the Limits of Automated Analysis of OpenTracing. International Symposium on Network Computing and Applications (IEEE NCA 2019) (The paper is waiting review).}
\end{itemize}

\section{Document Structure}
\label{sec:document_structure}

This section presents the document structure in this report, with a brief explanation of the contents in every section. The current document contains a total of seven chapters, including this one, Chapter~\ref{chap:introduction}~-~\nameref{chap:introduction}. The remaining six of them are presented as follows:

\begin{itemize}
    \item In Chapter~\ref{chap:methodology}~-~\nameref{chap:methodology} are presented the elements involved in this work, with their contributions, has well as the work plan, with ``foreseen'' and ``real'' work plans comparison and analysis.
    \item In Chapter~\ref{chap:state_of_the_art}~-~\nameref{chap:state_of_the_art} the current state of the field for this kind of problem is presented. This chapter is divided in three sections. The first one, Section~\ref{sec:concepts}~-~\nameref{sec:concepts} introduces the reader to the core concepts to know as a requirement for a full understanding of the topics discussed in this thesis. The second, Section~\ref{sec:technologies}~-~\nameref{sec:technologies} presents the result of a research for current technologies, that are able to help solving this problem and produce a proposed solution to be implemented. Finally, Section~\ref{sec:related_work}~-~\nameref{sec:related_work} presents the reader to related researches produced in the field of distributed tracing data handling.
    \item In Chapter~\ref{chap:research_objectives_and_approach}~-~\nameref{chap:research_objectives_and_approach} we present how we tackled this problem, the main difficulties that were found and the objectives involved to solve the issues that are presented. Also, in this chapter, a compilation of questions are presented and evaluated with some reasoning about possible ways to answer them.
    \item In Chapter~\ref{chap:possible_solution}~-~\nameref{chap:possible_solution} a possible solution for the presented problem is exposed and explained in detail. This chapter is divided in four sections. The first one, Section~\ref{sec:functional_requirements}~-~\nameref{sec:functional_requirements}, expose the functional requirements with their corresponding priority levels and a brief explanation to every single one of them. The second one, Section~\ref{sec:quality_attributes}~-~\nameref{sec:quality_attributes}, contains the gathered non-functional requirements that were used to build the solution architecture. The third one, Section~\ref{sec:technical_restrictions}~-~\nameref{sec:technical_restrictions}, presents the defined technical restrictions for this project. The last one, Section~\ref{sec:architecture}~-~\nameref{sec:architecture}, presents the possible solution architecture using some representational diagrams, and ends with an analysis and validation to check if the presented architecture meets up the restrictions involved in the architectural drivers.
    \item In Chapter~\ref{chap:implementation_process}~-~\nameref{chap:implementation_process}, the implementation process of the possible solution is presented with detail. This chapter is divided in three main sections covering the whole implementation process, from the input data set through the pair of components presented in the previous chapter. The first one, Section~\ref{sec:huawei_tracing_data_set}~-~\nameref{sec:huawei_tracing_data_set}, the tracing data set provided by Huawei to be used as the core data for research is exposed with some detail. Second, in Section~\ref{sec:open_tracing_processor_component}~-~\nameref{sec:open_tracing_processor_component} we present the possible solution for the first component, namely ``Graphy \gls{otp}'', that processes and extracts metrics from tracing data. The final Section~\ref{sec:data_analysis_component}~-~\nameref{sec:data_analysis_component} presents the possible solution for the second component, namely ``Data Analyser'', that handles data produced by the first component and produces the analysis reports. Also, in the last two sections presented, the used algorithms and methods in the implementations are properly detailed and explained.
    \item In Chapter~\ref{chap:results_analysis_and_limitations}~-~\nameref{chap:research_objectives_and_approach}, the gathered results, corresponding analysis and limitations of tracing data are presented. This chapter is divided in three main sections. The first one, Section~\ref{sec:anomaly_detection}~-~\nameref{sec:anomaly_detection}, the results regarding the gathered observations on the extracted metrics of anomalous service detection are presented and explained. Second, in Section~\ref{sec:trace_quality_analysis}~-~\nameref{sec:trace_quality_analysis} the results obtained from the quality analysis methods applied to the tracing data set are presented and explained. The final Section~\ref{sec:limitations_of_opentracing_data}~-~\nameref{sec:limitations_of_opentracing_data} we present the limitations felted when designing a solution to process tracing data, more precisely OpenTracing data.
    \item Last, in Chapter~\ref{chap:conclusions}~-~\nameref{chap:conclusions}, the main conclusions for this research work are presented. The chapter is divided in three main sections. First, Section~\ref{sec:brief_reflections}~-~\nameref{sec:brief_reflections}, a reflection about the implemented tools, methods produced and the open paths from this research are exposed. Also a reflection of the main difficulties felted with this research regarding the handling of tracing data are presented. Second, Section~\ref{sec:future_work}~-~\nameref{sec:future_work}, the future work that can be addressed considering this work is properly explained taking into consideration what is said in the previous section. Finally, Section~\ref{sec:concluding_research_questions}~-~\nameref{sec:concluding_research_questions}, the state of answers for the selected questions defined in this research are discussed.
\end{itemize}

Next, Chapter~\ref{chap:methodology} - \nameref{chap:methodology}, the elements involved in this work, their contributions and work plans for this research project are presented.

\checkoddpage
\ifthenelse{\boolean{oddpage}}
{ % Odd page
    \newpage
    \blankpage}
{ % Even page
}
%\glsresetall
\chapter{Methodology}
\label{chap:methodology}

The methodology of work carried out in this research project is presented in this chapter. First, every member involved will be mentioned as well as their individual contribution for the project. Second, the adopted approach and process organisation of the collaborators involved will be explained. Finally, the work plan as well as the work performed, including the foreseen and real work plans for the whole year of work are presented.

The main people involved in this project were myself, André Pascoal Bento, student at the Master course of Informatics Engineering at \gls{dei}, who carried out the investigation and development of the project. In second, Prof. Filipe Araújo, assistant professor at the University of Coimbra, who contributed with his vast knowledge and guidance on topics about distributed systems and cloud computing. In third, Prof. Jorge Cardoso, Chief Architect for Intelligent CloudOps at Huawei Technologies, who contributed with his vision, great contact with the topics addressed in this work and with the tracing data set from Huawei Cloud Platform~\cite{huawei_cloud_platform}. In fourth, Eng. Jaime Correia, doctoral student at \gls{dei}, who contributed with his vast technical knowledge regarding the topics of tracing and monitoring microservices.

\todo{CONTINUE FROM HERE!!!}

As this work stands for an investigation, it was necessary to perform an exploratory work and there were no clear development methodology adopted. Instead of a development methodology, meetings were scheduled in the beginning to happen every two weeks. In this meetings, the people mentioned in the previous paragraph were gathered together to discuss the work carried out in the last two weeks and topics like the information gathered, the analysis about some existing tools, ideas and solutions were discussed between all. In the end, although there wasn't defined some development methodology, the meeting that were carried out were more than enough to keep the productivity and good work.

For the work plan and starting by some numbers, the time spent in each semester of the year by week are sixteen hours for the first semester, and forty hours for the second one. In the end, it was spent a total of 304 hours for the first semester, starting in 11.09.2018 and ending in 21.01.2019 (19 weeks times 16 hours per week), and it is expected to be spent a total of 840 hours for the second semester, starting in 04.02.2019 and ending in 30.06.2019 (21 weeks times 40 hours per week).

In the beginning, there was a work plan for two semesters given in the project proposition. For record, these plans are presented in Figure~\ref{fig:proposed_work_plan_semester_1_and_2}.

For effects of analysis, the real work plan carried out in the first semester is presented in Figure~\ref{fig:real_work_plan_semester_1}.

\begin{figure}[H]
    \centering
    \includegraphics[width=1.00\textwidth]{images/proposed_work_plan_semester_1_and_2.pdf}
    \caption{Proposed work plan for the first and second semesters.}
    \label{fig:proposed_work_plan_semester_1_and_2}
\end{figure}

\begin{figure}[H]
    \centering
    \includegraphics[width=1.00\textwidth, height=0.25\textwidth]{images/real_work_plan_semester_1.pdf}
    \caption{Real work plan for the first semester.}
    \label{fig:real_work_plan_semester_1}
\end{figure}

As we can see, the ``foreseen'' work plan for the first semester has suffered some changes, when comparing it to the real work plan. The predicted task 1 - Study the state of the art(...), was branched into two 1 - Project Contextualisation and Background and 2 - State of the Art, and took some more time to do because of the non-concrete and lack of documentation in the technologies related to the subject of this thesis. The predicted task 2 - Integrate the existing work, was replaced by 3 - Prototyping and Technologies Hands-On. This replacement was done because of the interest in test the technologies gathered in the state of the art and see some results with them, enhancing our investigation work and allowing us to get a better visualisation of the data that we had back then. The remaining tasks took almost the predicted time to do.

Finally, it was generated a foreseen work plan for the second semester even knowing that, with almost one hundred percent of certainty the work plan will change when we perform the real work, it is presented in the figure \ref{fig:foreseen_work_plan_semester_2}. This work plan was generated taken into account the proposed work presented in the figure \ref{fig:proposed_work_plan_semester_2} and the effort needed to implement the defined solution presented in the chapter \ref{chap:possible_solution} - \nameref{chap:possible_solution}. To estimate the effort for each task we decided to, first group tasks by four groups regarding their complexity and work load, second discuss if they were in the right group taking into consideration what is defined in the solution and the task background knowledge, and third assign the defined values to each group. This values represent the work days for each task, and in this case we defined the following four: 3(three), 5(five), 8(eight) and 12(twelve) working days as they are akin to the Fibonacci suites\cite{project_estimation_times}. The only task that were not submitted to this, was the task 3 - Write the final report.

\begin{figure}[H]
    \centering
    \includegraphics[width=1.00\textwidth]{images/foreseen_work_plan_semester_2.pdf}
    \caption{Foreseen work plan for the second semester.}
    \label{fig:foreseen_work_plan_semester_2}
\end{figure}

\begin{figure}[H]
    \centering
    %\includegraphics[width=1.00\textwidth]{images/real_work_plan_semester_2.pdf}
    \caption{Real work plan for the second semester.}
    \label{fig:real_work_plan_semester_2}
\end{figure}

All the figures to expose the work plans have been created by an open-source tool called GanttProject\cite{gantt_project_tool} that produce Gantt charts, a kind of diagram used to illustrate the progress of the different stages of a project.

%-------------------------------------------------------------------------------------------------
\checkoddpage
\ifthenelse{\boolean{oddpage}}
{ % Odd page
\newpage
\blankpage}
{ % Even page
}
%-------------------------------------------------------------------------------------------------
\glsresetall
\chapter{State of the Art}
\label{chap:state_of_the_art}

In this chapter, we discuss the core concepts regarding the project, related work and the most modern techniques and available technology for the purpose today. All the information that will be presented was the result of a work of investigation through published articles, knowledge exchange and web searching.

The main purpose of the section \ref{sec:concepts} - \nameref{sec:concepts} is to introduce and provide a brief explanation about the core concepts. In the second section \ref{sec:related_work} - \nameref{sec:related_work} some published articles and posts of related work are presented. In the final section \ref{sec:technologies} - \nameref{sec:technologies}, all the important technologies are analysed and discussed using tables, diagrams and plain text.

\section{Concepts}
\label{sec:concepts}

The following concepts represents the baseline to understand the work related to this project.

\subsection{Microservices}
\label{subsec:microservices}

Microservices is ``an architectural style that structures an application as a collection of loosely coupled services, which implement business capabilities''\cite{microservices_definition}.

This kind of style has a very long history, and has being introduced and evolving since the first contact with topics like distributed computing, \gls{api} and containers.

The core concept of microservices stands in isolation, or by other words, what everyone wants to achieve when building a software with microservices in mind, is to share less things between the services and deal with correlated failures. In this sense, a service is a small part of the entire system (e.g. Get messages microservice), and represents a tiny feature of the whole service (e.g. Chat Service). To do this, normally every microservice is encapsulated inside a container (e.g. Docker container\cite{docker}), and each runs in its own process and communicates with the other using lightweight mechanisms, often an \gls{http} resource \gls{api}. ``A container is a standard unit of software that packages up code and all its dependencies so the application runs quickly and reliably from one computing environment to another''\cite{what_is_containers}.

On the other side, and for comparison purposes, we have another very well known architectural style, the monolithic. This style has a logically modular architecture, and the services are packaged and deployed in a single  application using a single code base. To compare both architectural styles presented before we have the figure \ref{fig:monolithic_and_microservices} that shows and provides a more clear insight about the differences between them.

\begin{figure}[H]
    \centering
    \includegraphics[width=1.00\textwidth]{monolithic_and_microservices.pdf}
    \caption{Monolithic and Microservices architectural styles\cite{microservices}.}
    \label{fig:monolithic_and_microservices}
\end{figure}

Every style presented has its owns pros and throwbacks and its usages benefit from case to case. A brief example of pros and throwbacks of each one is: for very large teams developing a big and complex service that needs to scale, its good to use the microservices architectural style, because they can tackle the problem of complexity by decomposing application into a set of manageable services which are much faster to develop by individual members, and therefore its much easier to understand and maintain, however its harder to assemble and perform the deployment of the whole service composed by tinny granular parts. In the monolithic architecture, it is normally simpler to deploy, because we  just have to copy a single packaged application to a server and run it, however when adding features to the application and it starts to grow in complexity, it gets harder to fully understand and made changes fast and correctly.

Microservices are simple enough to understand, but very hard to master in practice. Some people tend to think that microservices architecture reduce the complexity of the overall system over the monolithic architecture, however this isn't always like this. Communication is a big challenge in the realm of microservices and with a solution that integrates lots of interactions and connections throughout the whole system, the complexity starts to rise. Adding more microservices implies adding more complexity to the solution, and we have to be sure that new microservices can scale together with our existing ones.

``If you can’t manage building a monolith inside a single process, what makes you think putting network in the middle is going to help?'', Simon Brown\cite{simon_brown_microservices}.

Has been said before, the implementation of a microservices architecture raises the complexity of the solution, and we need to be aware of that when we design and implement a solution based on this architectural stile. On the other side, after having the system running up, the debbug process is a lot more complex too because, instead of having a single unit to analyze, we may have hundreds or even thousands of ``micro-units'' to trace and analyze. This kind of problem takes us to an important topic, the observation and control of the system performance, taking into consideration that this system is a microservice based system.

\subsection{Observability and Controlling Performance}
\label{subsec:observability_and_controlling_performance}

Observing is ``to be or become aware of, especially through careful and directed attention; to notice''\cite{observing_definition}.

Observability is an extension of observing and its definition is the following: ``Observability is to measure of how well internal states of a system can be inferred from knowledge of its external outputs''\cite{observability}.

The presented definitions represents the meaning of the words Observing and Observability are reflected exactly as it is in the project context. For example, observe the interaction between some microservices, regarding some data resulted from their interaction, to notice a certain fault in the whole/part of the system.

Controlling in control systems is ``to manage the behaviour of a certain system''\cite{control_systems}. Controlling and Observability are dual aspects of the same problem\cite{observability}.

When we want to understand the working and behaviour of a system, we need to watch it very closely and pay special attention to all details and data it provides. This kind of details and data may be in multiple structured text formats, and it can contain lots of information regarding the interaction between microservices and the corresponding access to them. The information generated by a microservices based system is normally represented as, what we call traces and spans, presented in the next subsection \ref{subsec:traces_and_spans} - \nameref{subsec:traces_and_spans}. Therefore, we may work with this information as a starting point to perceive the characteristics of the system and build a tool that is able to be aware of it and that can perform an analysis of the data to detect some system failures.

\subsection{Distributed Tracing}
\label{subsec:distributed_tracing}

\mytodos{// TODO: Write about Distributed Tracing: - Origin; - OpenTracing; - Specification; - Why; - How}

\subsection{Traces and Spans}
\label{subsec:traces_and_spans}

First things first, we can think in a trace as a group of spans. A trace is a representation of a data/execution path through the system and a span represents the logical unit of work in the system. A trace can also be a span. The span has an operation name, the start time of the operation, its duration and some annotations regarding the operation itself. An example of a span can be an \gls{http} call or a \gls{rpc} call. For a more clear insight of how spans are related with each other and with time, we have the figure \ref{fig:traces_and_spans_disposition_over_time}.

\begin{figure}[H]
    \centering
    \includegraphics[width=0.60\textwidth]{traces-and-spans.pdf}
    \caption{Traces and spans disposition over time.}
    \label{fig:traces_and_spans_disposition_over_time}
\end{figure}

As we can see in the figure \ref{fig:traces_and_spans_disposition_over_time}, the spans spread over time, overlapping each other, since nothing prevents the occurrence of multiple calls in very close times. In the same figure we can see some boxes, two represent traces (box A and B) and four represent spans (boxes B, C, D and E). For a brief example, and for this case we may think of the following cases to define each operation inherent to each box: A - ``Get user info'', B - ``Fetch user data from database'', C - ``Connect to MySQL server on 127.0.0.1'(10061)'', D - ``Can't connect to MySQL server on 127.0.0.1'(10061)'' and E - ``Send error result to client''. The creators of OpenTracing have made a data model specification that says, ``with a couple of spans, we might be able to generate a span tree and model a graph of a portion of the system''\cite{open_tracing_data_model_specification}. This is because they represents causal relationships in the system. As presented by the guys who defined this specification, and again for a more clear insight, the span tree can be like the one presented in the figure \ref{fig:span_tree_example}.

\begin{figure}[H]
    \centering
    \includegraphics[width=0.80\textwidth]{span_tree_example.pdf}
    \caption{Span Tree example.}
    \label{fig:span_tree_example}
\end{figure}

In the figure \ref{fig:span_tree_example} it's represented a span tree with a trace made up of eight spans. Every span must be a child of some other span, unless it is the root span. With the information presented in the span tree, we can generate a multi-directed graph of the system (explained in the subsection \ref{subsec:graphs} - \nameref{subsec:graphs}).

This type of data is extracted and can be obtained, as trace files or by transfer protocols ex. \gls{http}, from technologies like Kubernetes\cite{what_is_kubernetes}, OpenStack\cite{what_is_opensatck}, and other cloud or distributed management system technologies that implements some kind of system or code instrumentation using, for example, OpenTracing\cite{what_is_opentracing} or OpenCensus\cite{what_is_opencensus}.

In the end and as explained before, traces and spans contains some vital system details as they are the result of instrumentation of a part or the whole system and therefore, this kind of data can be used as a starting point resource information to analyse the system.

\subsection{Graphs}
\label{subsec:graphs}

As it was briefly explained  before, we might be able to model a graph of the system using a couple of spans. ``A Graph is a set of vertices and a collection of directed edges that each connects an ordered pair of vertices'' \cite{graph_standard_definition}.

Taking the very common sense of the term, a graph is an ordered pair G = (V, E), where G is the graph itself, V are the vertices/nodes and E are the edges. The figure \ref{fig:graph_visual_representation} gives us, a simple visual representation, of what a graph really is for a more clear understanding. There we can see a graph composed by a total of five nodes that contains some labels in it and, in this case, five relationships between them.

\begin{figure}[H]
    \centering
    \includegraphics[width=0.30\textwidth]{images/graph_visual_representation.pdf}
    \caption{Graph visual representation.}
    \label{fig:graph_visual_representation}
\end{figure}

This graph is the representation of the following information:

V = \{'G', 'R', 'A', 'P', 'H'\}

E = \{\{'G', 'R'\}, \{'R', 'A'\}, \{'A', 'P'\}, \{'P', 'H'\}, \{'H', 'G'\}\}

There are multiple types of graphs. In this term they can be undirected, where the set of edges don't have any orientation between a pair of nodes like in this example, or be directional, where the set of edges have one and only one orientation between a pair of nodes, or be a multigraph, where in multiple edges are more than one connection between a pair of node that represents the same relationship, and so forth.

Graphs can have many use cases, has they can model the representation of a lot of real life practical problems in the fields like physics, biology, social and information systems and for the purpose of this thesis, they are considered first class citizens.

\subsection{Graph Database}
\label{subsec:graph_database}

A \gls{gdb} is ``a database that uses graph structures for semantic queries with nodes, edges and properties to represent and store data''\cite{graph_database_definition}.

The composition of a \gls{gdb} is based on the mathematics graph theory, and therefore this databases uses three main components called nodes, edges, and properties. This main components are defined and explained in the following list:

\begin{itemize}
    \item Node: Are the entities in the graph. They can hold any number of attributes (key-value pairs) called properties. Nodes can be tagged with labels, representing their different roles in your domain. Node labels may also serve to attach metadata (such as index or constraint information) to certain nodes.
    \item Edge (or Relationships): provide directed, named, semantically-relevant connections between two node entities (e.g. André STUDIES\_IN \gls{dei}). A relationship always has a direction, a type, a start node, and an end node. Like a Node it can attach metadata. 
    \item Property: can be any kind of metadata attached to a certain Node or a certain Edge.
\end{itemize}

\subsection{Time Series Database}
\label{subsec:time_series_database}

A \gls{tsdb} is ``is a database optimised for time-stamped or time series data like arrays of numbers indexed by time (a date time or a date time range)''\cite{time_series_database_definition}.

This kind of databases are natively implemented using specialised database algorithms to enhance it's performance and efficiency due to the widely variance of access possible. The way this databases use to work on efficiency is to treat time as a discrete quantity rather than as a continuous mathematical dimension. Usually a \gls{tsdb} allows operations like create, enumerate, update, organise and destroy various time series entries.

The \gls{tsdb} and the \gls{gdb}, presented in this subsection and in the subsection before respectively, are at the time, the most wanted and fastest growing kind of databases due to their use cases in the trending fields of Cloud and Distributed based Systems and in the \textit{Internet of Things (IoT)}. The figure \ref{fig:fastest_growing_databases} presents the growing of this databases in the last two years. As we can see in the figure presented, the \gls{tsdb} and \gls{gdb} are distancing from the remaining databases in terms of popularity starting from the same spot in December of 2016. The predictions are that this databases will not stop increasing popularity, until this kind of systems described before start losing it too.  

\begin{figure}[H]
    \centering
    \includegraphics[width=1.0\textwidth]{images/popularity_of_time_series_databases.pdf}
    \caption{Fastest Growing Databases.\cite{time_series_databases_explained}}
    \label{fig:fastest_growing_databases}
\end{figure}

After presenting the core concepts for this work, we follow to the next section \ref{sec:related_work} - \nameref{sec:related_work} where some related work with this thesis is presented.

\section{Related Work}
\label{sec:related_work}

In this section are presented three section of the existing related work for tracing data analysis. After explaining each one, a brief reflection is made to note some directions of research for this project.

\subsection{Mastering AIOps}
\label{subsec:mastering_aiops}

\mytodos{// TODO: Explain what is done and how is done.}

\cite{mastering_aiops}

\subsection{Anomaly Detection using Zipkin Tracing Data}
\label{subsec:anomaly_detection_using_zipkin_tracing_data}

\mytodos{// TODO: Explain what is done and how is done.}

\cite{anomaly_detection_zipkin_tracing_data}

Wei Lee was contacted by email to understand better their research focus, however without success because no answer was given.

\subsection{Analyzing distributed trace data}
\label{subsec:analyzing_distributed_trace_data}

\mytodos{// TODO: Explain what is done and how is done.}

\cite{analysisng_distributed_trace_data}

\subsection{Research possible directions}
\label{subsec:research_possible_directions}

\mytodos{// TODO: Make a brief reflection about the related work.}

\section{Technologies}
\label{sec:technologies}

In this section are presented the technologies and tools that were researched, as well as the corresponding discussion considering the main objectives for this project. With the concepts presented in the section \ref{sec:concepts} in mind, the research were focused in a group of main topics regarding the problem we have in hands. The main topics were \ref{subsec:distributed_tracing_tools} - \nameref{subsec:distributed_tracing_tools}, \ref{subsec:graph_manipulation_and_processing_tools} - \nameref{subsec:graph_manipulation_and_processing_tools}, \ref{subsec:graph_database_tools} - \nameref{subsec:graph_database_tools} and \ref{subsec:time_series_database_tools} - \nameref{subsec:time_series_database_tools}.

\subsection{Distributed Tracing Tools}
\label{subsec:distributed_tracing_tools}

This sub-section presents what are the most used and known distributed tracing tools. This tools are mainly oriented for tracing distributed systems like microservices-based distributed systems. What they do is to fetch or receive trace data from this kind of complex systems, treat the information, and then present it to the user using charts and diagrams in order to explore the data in a more human-readable way. One of the best features presented in this tools is the possibility to perform queries on the tracing (e.g. by trace id and by time-frame). The table \ref{table:tracing_tools} presents the most well-known OpenSource tracing tools.

\begin{table}[H]
\caption{Tracing tools comparison.}
\label{table:tracing_tools}
\centering
\large
\begin{tabularx}{\linewidth} {
    |>{\hsize=0.70\hsize}X| 
     >{\hsize=1.15\hsize}X|
     >{\hsize=1.15\hsize}X| }
     \hline
    \textbf{Name} 
    & Jaeger
    & Zipkin \\ \hline
    \textbf{Repository}
    & Jaeger GitHub \cite{jaeger_github}
    & Zipkin GitHub \cite{zipkin_github} \\ \hline
    \textbf{Brief description}
    & It is a distributed tracing and monitoring system released as open source by Uber Technologies, that is used for monitoring and troubleshooting microservices-based distributed systems.
    & It is a distributed tracing and monitoring system. It helps gather timing data needed to troubleshoot latency problems in microservice architectures. It manages both the collection and lookup of this data. Zipkin’s design is based on the Google Dapper paper. \\ \hline
    \textbf{Pros}
    & OpenSource. \newline
    Docker-ready. \newline
    Can be used with some Zipkin functionalities, as it has a collector dedicated to it. \newline
    Dynamic sampling rate. \newline
    Browser UI.
    & OpenSource. \newline
    Docker-ready. \newline
    Allows lots of span transport ways (HTTP, Kafka, Scribe, AMQP). \newline
    Browser UI. \\ \hline
    \textbf{Cons} 
    & Only supports two span transport ways (UDP and HTTP).
    & Fixed sampling rate. \\ \hline
    \textbf{Used mainly by}
    & Uber
    & Lightstep \\ \hline
\end{tabularx}
\end{table}

As we can see, this kind of tools are very similar and very good for monitoring and tracing a system as they provide a bunch of pros like being opensource, dockerized, support for some well known technologies for span transport and aggregate the spans in a good representational browser user-interface. However they are always focused on span and trace lookup and presentation, and do not provide a more interesting analysis of the system, for example to determine if there is any problem related to some microservice presented in the system. This kind of work falls into the user (\gls{devops}) and he needs to perform the investigation and analyse the traces and spans with the objective of find anything wrong with them.

This kind of tools can be a good starting point for the problem that we face, because they already do some work for us like grouping the data generated by the system and provide some way to visualize it. 

\subsection{Graph Manipulation and Processing Tools}
\label{subsec:graph_manipulation_and_processing_tools}

Considering that we have data to be processed and manipulated, we have to be sure that we can handle it for analysis. For this purpose, and knowing that the data is a representation and an abstraction of a graph, we needed to study the frameworks available this task. The table \ref{table:graph_manipulation_and_processing_tools_comparison} presents the main technologies available at the time for graph manipulation and processing.


\begin{table}[H]
\caption{Graph manipulation and processing tools comparison.}
\label{table:graph_manipulation_and_processing_tools_comparison}
\centering
\large
\begin{tabularx}{\linewidth} {
    |>{\hsize=0.7\hsize}X| 
     >{\hsize=1.1\hsize}X|
     >{\hsize=1.1\hsize}X| 
     >{\hsize=1.1\hsize}X| }
\hline
\textbf{Name} 
& Apache Giraph \cite{apache_giraph}
& Ligra \cite{ligra_graph_processing_framework}
& NetworkX \cite{networkx} \\ \hline
\textbf{Description}
& It's an iterative graph processing system built for high scalability. For example, it is currently used at Facebook to analyse the social graph formed by users and their connections.
& A library collection for graph creation and manipulation, and for analysing networks. 
& A Python package for the creation, manipulation, and study of the structure, dynamics, and functions of complex networks. \\ \hline
\textbf{Licence}\cite{software_license}
& Free Apache 2
& MIT
& BSD - New License \\ \hline
\textbf{Supported languages} 
& Java and Scala.
& C and C++. 
& Python. \\ \hline
\textbf{Pros} 
& Distributed and very scalable. \newline
Excellent performance (Can process one trillion edges using 200 modest machines in 4 minutes).
& Can handle very large graphs. \newline
Exploit large memory and multi-core \gls{cpu}'s (Vertically scalable).
& Good support and very easy to install with Python. \newline
Lots of graph algorithms already implemented and tested. \newline
Mature project.\\ \hline
\textbf{Cons} 
& Uses the ``Think-Like-a-Vertex'' programming model that often forces into using sub-optimal algorithms and is quite limited and sacrifices performance for scaling out. \newline
Can't perform many complex graph analysis tasks because it primarily supports Bulk synchronous parallel.
& Lack of documentation and therefore, very hard to use. \newline
Don't have many usage in the community.
& Not scalable (single-machine). \newline
High learning curve due to the maturity of the project. \newline
Begins to slow down when there's a high amount of data (400.000 plus nodes). \\ \hline
\end{tabularx}
\end{table}

With the information presented in the previous table, we can have a notion that this three frameworks don't work and perform in the same level in many ways. 

One thing to consider when comparing them is the scalability and performance that each can provide, for instance, in this component the first one, Apache Giraph is the winner since it is implemented with the distributed systems paradigm in mind and can scale to multiple-machines to get the job done in no time, considering high amounts of data. In other way, we have the third framework, called NetworkX, that is different from the previous as it works in a single-machine and doesn't have the ability to scale to multiple-machines. This can be a very problematic feature if we are dealing with very high amounts of data and we have to process it in short amounts of time. The last framework, called Ligra, works in a single-machine environment like the previous one, but it can scale vertically as it has the benefit of use and exploit multi-core \gls{cpu}'s. 

The second, and also most important thing to consider, is the support and quantity of implemented graph algorithms in the framework, and in this field the tables turned, and the NetworkX has a lot of advantage as it have lots of implemented graph algorithms defined and studied in graph and networking theory. The remaining frameworks don't have very support either because they don't have it documented or because the implementation and architecture that where considered don't allow to implement it.

For a more clear insight of the position of the presented technologies in the previous table, we have the figure \ref{fig:graph_manipulation_and_performance_tools_diagram_comparison} \cite{graph_data_management_systems}.

\begin{figure}[H]
    \centering
    \includegraphics[width=0.80\textwidth]{images/graph_manipulation_tools_diagram_comparison.pdf}
    \caption{Graph manipulation tools comparison, regarding scalability and graph algorithms.}
    \label{fig:graph_manipulation_and_performance_tools_diagram_comparison}
\end{figure}

With the presented figure, our perception of what we might choose when considering this tools is more clear, but we have always some trade offs we cannot avoid. The best approach, and if it's possible, is to consider the usage of an hybrid environment where Giraph and NetworkX coexist one with another, as one fills the gaps of the other, but always taking into consideration that a bottleneck will occur between them \cite{graph_frameworks_performance_evaluation} and that there are almost none implementation where they coexist \cite{graph_frameworks_performance_evaluation} because of their disparity.

\subsection{Graph Database Tools}
\label{subsec:graph_database_tools}

Manipulating and process graph data is not enough, we need to store this data somewhere, and to do this we need a \gls{gdb}. The results of the research for the best graph databases tools available are presented in the table \ref{table:graph_databases_comparison}.

\begin{table}[!b]
\caption{Graph databases comparison.}
\label{table:graph_databases_comparison}
\centering
\large
\begin{tabularx}{\linewidth} {
    |>{\hsize=0.7\hsize}X| 
     >{\hsize=1.1\hsize}X|
     >{\hsize=1.1\hsize}X| 
     >{\hsize=1.1\hsize}X| }
    \hline
    \textbf{Name} 
    & ArangoDB \cite{arangodb_documentation}
    & Facebook TAO \cite{facebook_tao_article}
    & Neo4J \cite{neo4j_documentation} \\ \hline
    \textbf{Description} 
    & It's a NoSQL database developed by ArangoDB Inc. that uses a proper query language to access the database.
    & TAO, “The Associations and Objects”, is a proprietary database, developed by Facebook, that stores all the data related to the users in the social network. 
    & It's the most popular open source graph database. Has been developed by Neo4J Inc. and is completely open to the community. \\ \hline
    \textbf{Licence}
    & Free Apache 2 
    & Proprietary 
    & GPLv3 CE \\ \hline
    \textbf{Supported languages} 
    & C++ \newline 
    Go \newline
    Java \newline
    JavaScript \newline
    Python \newline
    Scala
    & \centering -----
    & Java \newline
    JavaScript \newline
    Python \newline
    Scala \\ \hline
    \textbf{Pros} 
    & Multi data-type support (key/value, documents, graphs). Allows the combination of different data access patterns in a single query. Supports cluster deployment. 
    & Very fast(~=100ms latency). Accepts millions of calls per second. Distributed.
    & Supports ACID(Atomicity, Consistency, Isolation, Durability)\cite{acid_definition}. High-availability. Has a visual node-link graph explorer. REST \gls{api} interface. Most popular open source graph database. \\ \hline
    \textbf{Cons} 
    & Needed to learn a new query language called AQL(Arango Query Language). High learning curve. Has paid version with high price tag.
    & Not accessible to use.
    & Can’t be distributed (It needs to be vertically scaled). \\ \hline
\end{tabularx}
\end{table}

As we can notice by the data provided by the presented table, the state of the art about graph databases is not very good. The offers are very limited and all of them lack something when we start to see them in detail. The interest in this databases is increasing as graph technology tend to have many use cases and solve lots of problems nowadays.

Facebook detains the most powerful and robust system for this purpose, but as it is the base of their business because they need to perform large operations in their huge social graph in reduced times, it is a proprietary technology and is only referenced in some articles \cite{facebook_tao_article}.

The remaining two tools, are very supported by the community because of their license and demand, however based on the stars and forks of their repositories, Neo4J is more well received by the community and tends to become more popular. It doesn't implement horizontal scalability by design and this can be a risk when using it in systems with scalabilty in mind, but there are some authors that report they were able to perform implementations and surpass the scalability issue, however with many snags\cite{neo4j_scalable}. ArangoDB supports scalability by default as we can see in the figure \ref{fig:arangodb_vs_neo4j_scalability}\cite{arangodb_vs_ne4j}, but it has a very hard query language with a high learning curve inherent to it, and it is payed to use some special features like SmartGraphs storage\cite{arangodb_smart_graphs} that improves the writing of graph in distributed databases.

\begin{figure}[H]
    \centering
    \includegraphics[width=0.80\textwidth]{images/arangodb_vs_neo4j_scalability.pdf}
    \caption{ArangoDB vs. Neo4J scalability over complexity.}
    \label{fig:arangodb_vs_neo4j_scalability}
\end{figure}

\subsection{Time-Series Database Tools}
\label{subsec:time_series_database_tools}

As we intend to extract useful data from span trees and graphs, we need to store it somewhere. We already know that the spans and trace data are directly related with time based information, explained in the subsection \ref{subsec:traces_and_spans} - \nameref{subsec:traces_and_spans}, so the best way to store the gathered or calculated information from them is in a \gls{tsdb}.

The figure \ref{fig:time_series_databases_ranking} present the ranking of the \gls{tsdb} at the current time\cite{tsdb_ranking}.

\begin{figure}[h!]
    \centering
    \includegraphics[width=1.00\textwidth]{images/time_series_databases_ranking.pdf}
    \caption{\gls{tsdb}s ranking from 2013 to 2019.}
    \label{fig:time_series_databases_ranking}
\end{figure}

The table \ref{table:time_series_databases_comparison} exposes a comparison between the two top databases presented in the ranking, the \textit{InfluxDb} and \textit{OpenTSDB}, two very well known databases in the world of \gls{tsdb}, in order to understand the advantages and disadvantages of each one.

\begin{table}[H]
\caption{Time-series databases comparison.}
\label{table:time_series_databases_comparison}
\centering
\large
\begin{tabularx}{\linewidth} {
    |>{\hsize=0.50\hsize}X| 
     >{\hsize=1.25\hsize}X|
     >{\hsize=1.25\hsize}X| }
    \hline
    \textbf{Name} 
    & InfluxDB \cite{influxdb}
    & OpenTSDB \cite{opentsdb} \\ \hline
    \textbf{Description} 
    & It is an open-source time series database developed by InfluxData written in Go and optimised for fast, high-availability storage and retrieval of time series data in fields such as operations monitoring, application metrics, Internet of Things sensor data, and real-time analytics.
    & It is a distributed, scalable Time Series Database (TSDB) written on top of HBase. OpenTSDB was written to address a common need: store, index and serve metrics collected from computer systems (network gear, operating systems, applications) at a large scale, and make this data easily accessible and graphable. \\ \hline
    \textbf{Licence}
    & MIT 
    & GPL \\ \hline
    \textbf{Supported languages} 
    & Erlang \newline
    Go \newline
    Java \newline
    JavaScript \newline
    Lisp \newline
    Python \newline
    R \newline
    Scala
    & Erlang \newline
    Go \newline
    Java \newline
    Python \newline
    R \newline
    Ruby \\ \hline
    \textbf{Pros} 
    & Scalable in the enterprise version. \newline
    Outstanding high performance. \newline
    Accepts data via HTTP, TCP, and UDP protocols. \newline
    SQL like query language. \newline
    Allows real-time analytics.
    & It's massively scalable. \newline
    Great for large amounts of time-based events or logs. \newline
    Accepst data via HTTP and TCP access protocols. \newline
    Good platform for future analytical research into particular aggregations on event/log data. \newline
    Doesn't have paid version. \\ \hline
    \textbf{Cons} 
    & Enterprise high price tag. \newline
    Clustering support only available in the enterprise version.
    & Expensive to try. \newline
    Not a good choice for general-purpose application data. \\ \hline
\end{tabularx}
\end{table}

Based on the information presented in the referenced table, we can notice that this two databases are very similar on what they offer like the access protocols and scalability capabilities. In the point of licence, both are open source, however the first one, InfluxDB, has an enterprise paid version that is not very well exposed in its documentations and much people don't even notice it, contrarily to OpenTSDB which is completely free. The enterprise version of InfluxDB provides clustering support, high availability and scalability\cite{influxdb_vs_opentsdb}, features that OpenTSDB offer for free, however in terms of performance, InfluxDB outperforms OpenTSDB in almost every benchmark by a far distance as we can see in the figures \ref{fig:influxdb_vs_opentsdb_write_throughput} and \ref{fig:influxdb_vs_opentsdb_storage_requirements}.

\begin{figure}[H]
    \centering
    \includegraphics[width=1.00\textwidth]{images/influxdb_vs_opentsdb_write_throughput.pdf}
    \caption{InfluxDB vs OpenTSDB write throughput performance\cite{influxdb_vs_opentsdb}.}
    \label{fig:influxdb_vs_opentsdb_write_throughput}
\end{figure}

\begin{figure}[H]
    \centering
    \includegraphics[width=1.00\textwidth]{images/influxdb_vs_opentsdb_storage_requirements.pdf}
    \caption{InfluxDB vs OpenTSDB storage requirements\cite{influxdb_vs_opentsdb}.}
    \label{fig:influxdb_vs_opentsdb_storage_requirements}
\end{figure}

After providing the state of the art involved with this work to the reader in this chapter, the document flows to the next chapter \ref{chap:research_objectives_and_approach} - \nameref{chap:research_objectives_and_approach}, where the objectives of this research and the approach taken to each component of work are presented and discussed.

%-------------------------------------------------------------------------------------------------
\checkoddpage
\ifthenelse{\boolean{oddpage}}
{ % Odd page
\newpage
\blankpage}
{ % Even page
}
%-------------------------------------------------------------------------------------------------
\glsresetall
% Research Objectives and Approach ---------------------------------------------------------------
\chapter{Research Objectives and Approach}
\label{chap:research_objectives_and_approach}

This chapter presents the research objectives and approach used in this thesis. We will start to discuss how we faced the problem, and what were the main difficulties that we found when handling this kind of problem as well as the ways we have taken to deal with it.

The debugging process in distributed systems and microservice based systems is not an easy task to perform, because of the way the system is designed using this kind of architecture style, as explained in the subsection \ref{subsec:microservices} - \nameref{subsec:microservices}. Reasoning about concurrent activities of system nodes and even understanding the system's communication topology can be very difficult. A standard approach to gaining insight into system activity is to analyze system logs, but this task can be very tedious and complex process. The main existing tools are the ones presented in \ref{subsec:monitoring_and_tracing_tools}, but they only do the job of gather and present the information to the user in a more gracefully way, however they rely on the user perception to do the search to find issues that exist in their platform by performing queries to the spans and trace data. So, with this problem ahead, we started looking for the needs of Sysadmins and/or \gls{devops} when they wanted to scan and analyse their system searching for issues.

To do this and narrow the problem we were facing, we decided to talk with some \gls{devops} personal and expose them the situation, with the objective to gather their main needs and ideas in mind, we putted ourselves in their perspective when talking to them to try and find what are the main difficulties when they perform their search for issues in the system in a \textit{``As a \gls{devops} i want to...''} situation. The kind of questions that were placed were like: \textit{``What are the most common issues?''}, \textit{``What are the variables involved in this kind of issues?''} and \textit{``What are the correlations between this variables and the most common issues?''}. From this discussions and conversations emerged the following eight core questions:

\begin{itemize}
    \item[\textbf{1.}] What is the neighbourhood of one service?
    \item[\textbf{2.}] Is there any problem (Which are the associated heuristics)?
    \item[\textbf{3.}] Is there any faults related to the system design/architecture?
    \item[\textbf{4.}] What is the root problem, when A, B, C services are slow?
    \item[\textbf{5.}] How are the requests coming from the client?
    \item[\textbf{6.}] How endpoints orders distributions are done?
    \item[\textbf{7.}] What is the behaviour of the instances?
    \item[\textbf{8.}] What is the length of each queue in a service?
\end{itemize}

The next step was to work on the questions presented above. We decided to split them in more concise questions, refine and filter the most relevant to define our objective, and after that, check with someone involved in the \gls{devops} field if the final questions represent some of their needs. First, to handle the information presented in this eight initial questions, we decided to create what we called a ``Project Questions Board''. This board consists on a Kanban \cite{kanban_board} style board present in the project git repository, were everyone involved in the project could access and modify it. The board was defined with four lanes: ``To refine'', ``Interesting'', ``Refined'' and ``Final Questions'', and the process were to cycle the questions through every lane, generating new ones and filtering others. After this, and to check if the final questions were really some that represented the needs of a \gls{devops}, some colleagues that work directly in the field were contacted and the questions were exposed to them. In the end, the ten questions that were produced in the final lane represented right what are some of their needs. The final questions, their corresponding description\textbf{(D)} and explanation of the expected work\textbf{(W)} that must be performed to each one are exposed bellow:

\begin{itemize}
    \item[\textbf{1.}] What is the neighbourhood of one service, based on incoming requests? \begin{itemize}
        \item[D.] The neighbourhood of one service is a very important information to know due to the simple fact that it represents the interaction between the microservices. The incoming requests can map the interactions between the microservices and with this kind of information, we can check and analyse the service dependencies.
        \item[W.] Implies generate a graph, based on the spans and traces, using the outgoing connections, from a certain node, that are correlated with the incoming connection(s).
    \end{itemize}

    \item[\textbf{2.}] What is the neighbourhood of one service, based on outgoing requests?
    \begin{itemize}
        \item[D.] Similar to the previous question, but this time, instead of incoming requests we focus on the outgoing requests of one service. 
        \item[W.] Implies generate a graph, based on the spans and traces, using the incoming connections, from a certain node, that are correlated with the outgoing connection(s).
    \end{itemize}

    \item[\textbf{3.}] How endpoints orders distributions are done, when using a specific endpoint?
    \begin{itemize}
        \item[D.] The distributions of microservices in a system allows us to understand if a certain endpoint groups with others or if it is an isolated service, which stands for its relevance to the whole system.
        \item[W.] Implies generate a graph, based on the spans and traces, then calculate the degree of a certain node that represents the endpoint, to finally check if it is an isolated, a leaf or a dominating (high or low depending on the degree of the other degrees) endpoint.
    \end{itemize}
    
    \item[\textbf{4.}] How requests are being handled by a specific endpoint?
    \begin{itemize}
        \item[D.] This question has the objective of analyse the status of the requests that arrive or depart from a specific endpoint. This status represents if the requests was well succeed or not.
        \item[W.] Implies to analyse the data from the requests that pass through a specific endpoint. Based on the annotations presented in the spans and traces, we are able to check if the requests are resulting in success or in error. 
    \end{itemize}

    \item[\textbf{5.}] Which endpoints are the most popular?
    \begin{itemize}
        \item[D.] The popularity of a certain endpoint is very important because it represents the importance of this endpoint to the system.
        \item[W.] Implies to retrieve the most popular service, based on the spans and traces, and get the services with more incoming connections sorted by the number of incoming connections.
    \end{itemize}

    \item[\textbf{6.}] Is there any problem related to the response time?
    \begin{itemize}
        \item[D.] Response time is must watch variable because, for example, strange high values may represent a problem in the system performance.
        \item[W.] Implies to get the response time of every trace (difference between end and start time of every span in the trace) and then calculate and store some measurements like the average time, the maximum time, the minimum time and variance. After having some stored values, the system must perform calculations and check if there is too much disparity between them to determine if there is a problem in the response time.
    \end{itemize}

    \item[\textbf{7.}] Is there any problem related to the morphology?
    \begin{itemize}
        \item[D.] The morphology of the system allows us to understand if some endpoints are common in the system (they usually exist), or if they only are instantiated in specific situations.
        \item[W.] Implies generate multiple graphs, based on a certain group of spans and traces that are contained in a certain time interval. Then we need to store the graphs gradually using some graph storing mechanism to perform the difference of subsequent stored graphs. This result of the difference between graphs must be stored in a time-series storing mechanism, to be accessed later and determine if there were hard changes that could lead to morphology problems in the system thought time.
    \end{itemize}
    
    \item[\textbf{8.}] Is there any problem related to the entire workflow of (one or more) requests?
    \begin{itemize}
        \item[D.] Analyse the request workflow through the system is a good practice, as it represents the interaction triggered by the request in the system and its resulting behaviour. This can lead to find out if the system has cycles and if they are normal or represent a problem to solve.  
        \item[W.] Implies to generate the graph of the system, identify the path of some request(s) in the system and then perform the calculation to verify and identify if there were cycles presented in the graph involved in the path of the request(s). The results of this calculations must be stored in a time-series storing mechanism, to be accessed later and determine if this cycles are normal, or if they represent a problem related to the request(s) work-flow, based on the kind of request.
    \end{itemize}

    \item[\textbf{9.}] Is there any problem related to the occupation/load of a specific endpoint?
    \begin{itemize}
        \item[D.] The occupation/load of a specific endpoint in this case is represented by the number of requests in queue of a specific endpoint. This value is very important because it represents if the endpoint is in overflow or not.
        \item[W.] Implies to get the number of requests in queue of a specific endpoint and then calculate and store some measurements like the average, maximum, minimum and variance of the number of requests in queue. After having some stored values, the system must perform calculations and check if there is too much disparity between them to determine if there is a problem in the occupation/load.
    \end{itemize}

    \item[\textbf{10.}] Is there any problem, related to the number/profile of the client requests?
    \begin{itemize}
        \item[D.] The number of client requests and their corresponding profile represents the behaviour of clients when using the system. This can be used to identify problems of bad system usage from the clients, like for example a DDoS (Denial Of Service).
        \item[W.] Implies calculate the number of accesses to the system, based on the spans and traces annotated with client requests of a certain time interval, and store the calculations for every node. After having some stored values, the system must determine the level regions of access based in the available data (profile of requests, ex.: high, moderate and low), and check if there were to much requests outside of the defined level regions.
    \end{itemize}
\end{itemize}

 These final questions, with a slight reformulation, could be exported to high level of abstraction functional requirements of the monitoring tool that we want to develop. After having this questions, we decided to study the current state of art to check how things are done nowadays regarding this subject and we found that some tools perform the process of convert spans and traces to a graph, that represents the system at that current time interval, however they do not perform any kind of analysis and study over the span tree and the graph after that\cite{spans_analysis}.
 
 Considering this, what we decided to do was to develop a simple prototype tool to test some state of the art tools. What we were able to achieve was to do the reconstruction of the graph, using our own data (this data was provided by Prof. Jorge Cardoso, representing an approximate two hour collection of spans and traces, about 400.000 spans, generated by one of their clusters). At this point, and since we have already held the hands-on of some tools at the moment, we were ready to start and think about the solution we need to build. Therefore, we decided to specify the solution, and considered to build a monitoring tool named by ourselves, \textit{Graphy}.
 
 In a very briefly explanation, we want that \textit{Graphy} be able to calculate relevant metrics from the span trees and the generated graphs, and to work with this kind of metrics to perform the system analysis and answer the questions exposed above. To perform some of this work, it will be resourcing to machine learning algorithms that we will need to study in parallel with the implementation, as we cannot predict what we might encounter when retrieving the metrics at the time. The machine learning algorithms are to process the metrics and perform some deductions regarding the system behaviour over the time.
%-------------------------------------------------------------------------------------------------


%-------------------------------------------------------------------------------------------------
\checkoddpage
\ifthenelse{\boolean{oddpage}}
{ % Odd page
\newpage
\blankpage}
{ % Even page
}
%-------------------------------------------------------------------------------------------------
\glsresetall
\chapter{Proposed Solution}
\label{chap:proposed_solution}

In this Chapter, we present and discuss a possible solution to be implemented regarding the main problem to solve in this research, the data to process and the research questions to be answered. To present the solution and explain it, we will cover some aspects considered when defining a software based solution. This topics are: functional requirements~\ref{sec:functional_requirements}, quality attributes (non-functional requirements)~\ref{sec:quality_attributes} and finally, the architecture~\ref{sec:architecture} produced based on all previous topics.

The starting point for our proposed solution is the tracing data provided by Huawei. Tracing must be ingested by an entry component, capable of extracting metrics from tracing data. The outcome of this module are metrics and metadata in files to be further processed by a second component. This second component has the duty of analysing the output data from the first module, and point out service anomalies.

For a clear insight about our solution, the proposed approach in high level of abstraction is presented in the Figure~\ref{fig:proposed_approach}.

\begin{figure}[H]
    \centering
    \includegraphics[width=1.00\textwidth]{images/proposed_solution.pdf}
    \caption{Proposed approach.}
    \label{fig:proposed_approach}
\end{figure}

Figure~\ref{fig:proposed_approach} shows the proposed process order for tracing data. We expect to have two main components, one for data extraction and another for data analysis. The input for each are tracing data and processed data from the first component respectively. The outcome is to answer the research questions defined in Section~\ref{sec:research_questions}~-~\nameref{sec:research_questions}.

Next Section~\ref{sec:functional_requirements}~-~\nameref{sec:functional_requirements} covers the functional requirements for this solution.

\section{Functional Requirements}
\label{sec:functional_requirements}

In software engineering, functional requirements defines the intended function of a system and its components. To present the functional requirements for our solution proposition, an id, the corresponding name and its priority are provided. The notation used in priority was based on the urgency that we expected from feature implementation. Three priority levels were used: High, Medium and Low. Therefore, the functional requirements for the proposed solution, sorted by priority levels, are presented in Table~\ref{table:functional_requirements_specification}.

\begin{table}[H]
    \caption{Functional requirements specification.}
    \label{table:functional_requirements_specification}
    \centering
    \begin{tabularx}{\linewidth} {
            |>{\hsize=0.10\hsize}X|
            >{\hsize=0.75\hsize}X|
            >{\hsize=0.15\hsize}X|}
        \cline{1-3}
         \textbf{ID}
         & \textbf{Name}
         & \textbf{Priority}                                                                                                                                                                                  \\ \hline \hline
         FR-1
         & The system must be able to ingest tracing data from a files or external distributed tracing tools.
         & High \\ \hline
         FR-2
         & The system must be able to retrieve service dependency graphs from distributed tracing tools.
         & High \\ \hline
         FR-3
         & The system must be able to store service dependency graphs in a graph database.
         & High \\ \hline
         FR-4
         & The system must be able to store time-series metrics extracted from tracing data in a time-series database.
         & High \\ \hline
         FR-5
         & The system must be able to extract the number of calls per service (total, incoming and outgoing) from tracing data.
         & Medium \\ \hline
         FR-6
         & The system must be able to extract the response time per service from tracing data.
         & Medium \\ \hline
         FR-7
         & The system must be able to generate request work-flow paths from tracing data.
         & Medium \\ \hline
         FR-8
         & The system must be able to generate request work-flow paths from tracing data.
         & Medium \\ \hline
         FR-9
         & The system must be able to calculate request ratio of success and error, for specific services, from tracing data.
         & Medium \\ \hline
         FR-10
         & The system must be able to calculate the degree of services from service dependency graphs.
         & Medium \\ \hline
         FR-11
         & The system must be able to retrieve the difference between two service dependency graphs.
         & Medium \\ \hline
         FR-12
         & The system must be able to produce a report about spans structure using a defined OpenTracing structural schema.
         & Low \\ \hline
         FR-13
         & The system must be able to calculate the time coverage of traces in a given time-frame.
         & Low \\ \hline
         FR-14
         & The system must be able to identify regions of outliers presented in multiple time-series.
         & Low \\ \hline
    \end{tabularx}
\end{table}

Functional requirements defined in Table~\ref{sec:functional_requirements} were written based on defined research questions presented in Section~\ref{sec:research_questions}. These functional requirements can be grouped in three groups due to their priority levels. The first four (FR-1 to FR-4) are presented with high level of priority, because they represent the base functionality needed to implement the remaining requirements. The next eight functional requirements (FR-5 to FR-11), are time based metric extraction from tracing. The remaining three (FR-12 to FR-14) are related with trace testing and anomaly detection based in time-series thus the low priority.

Next Section~\ref{sec:quality_attributes}~-~\nameref{sec:quality_attributes} covers the proposed approach non-functional requirements.

\section{Quality Attributes}
\label{sec:quality_attributes}

Another important consideration, when designing a software system, is to specify all the quality attributes (also called non-functional requirements). These type of requirements are usually Architecturally Significant Requirements and are the ones that require more from software architect's attention, as they reflect directly all architecture decisions. To specify them, a representation called utility tree is often used. In this tree, the \gls{qa} are placed by an order of priority considering their impact for the architecture and for the business. The priority codification for the \gls{qa} is:
%  in order to consider the trade-offs and decide the weight of each in the produced architecture.

\begin{itemize}
    \item H. High
    \item M. Medium
    \item L. Low
\end{itemize}

To describe them properly, six important aspects must be included in \gls{qa} definition: \emph{stimulus source}, \emph{stimulus}, \emph{environment}, \emph{artefact}, \emph{response} and \emph{measure of the response}.

Figure~\ref{fig:utility_tree} contains all raised \gls{qa} for this proposed solution exposed in an utility tree structure, sorted alphabetically by their general \gls{qa} name, and after by the architectural impact pair (Architecture and Business).

\begin{figure}[H]
    \centering
    \includegraphics[width=1.00\textwidth]{images/utility_tree.pdf}
    \caption{\gls{qa} utility tree.}
    \label{fig:utility_tree}
\end{figure}

Figure~\ref{fig:utility_tree} shows us that only two Interoperability \gls{qa} were defined. An explanation for both is provided bellow:

\begin{itemize}
    \item[\textbf{QA1}] (Interoperability): Since the proposed solution must ingest tracing data. This information is usually found in distributed tracing tools already used by operators. To gather this information, access to an external distributed tracing tool is an important feature.As this is considered the starting point to obtain our data, we considered a Medium level for the architecture and a Low for the business.

    \item[\textbf{QA2}] (Interoperability): Since the proposed solution will be accessing an external distributed tracing system or outputs generated by it, all interactions with these systems must not cause conflicts. This is very important in the business perspective, because if our solution is not co-habitable with already used systems, it may be completely rejected. For the architectural perspective it does not represent a big impact, and therefore a Low level was assigned.

    %\item[\textbf{QA3}] (Performance): We defined this \gls{qa} taking into account the number of spans produced in an hour, by the system were we gathered our data. As it produces approximately 200.000 spans in an hour, and to ease our research work when using the tool, we decided to set the target for our solution as 1.000.000 spans in about a minute. This \gls{qa} will have an high architectural impact, as it can define a certain technology for graph processing. For the business perspective, we considered a Medium level, as it presents some interest.

    %\item[\textbf{QA4}] (Scalability): Due to the amount of data needed to process and store over time, our system has to be able to store the data into multiple machines because it may start running out of space. We decided to give a medium level of architectural impact as it can change the solution in terms of storage components. For the business this has an high impact, as it need more machines to run the solution if this \gls{qa} is fully considered against the remaining.

    %\item[\textbf{QA5}] (Traceability): This \gls{qa} was considered due to the simple fact that we need to bee able to see what's the system is doing, when it's processing the data. For this \gls{qa} we decided to give low levels for both architecture and business, as it does not represent relevance to any of them.

    %\item[\textbf{QA6}] (Testability): As the system will be analysing external systems, we need to be sure that it analyses it in a correct way, and to be sure of this we need to test our solution. We considered a medium level for the architectural impact for this \gls{qa}, because its implementation needs to analyse some components from within the system. For the business we considered a low level, because the main interest to test the system is ours in order check if it is working correctly.
\end{itemize}

\section{Technical Restrictions}
\label{sec:technical_restrictions}

In this Section, technical restrictions considered in proposed solution are presented.

In software engineering, after specifying functional and non-functional requirements for a solution, comes the specification of business restrictions, however, in this project none were raised due to the fact that this work is focused on exploration and research.
%, and that there isn't any formal client defined.

To define the technical restrictions, we used an id and its corresponding description. Table~\ref{table:technical_restrictions_specification} presents the technical restrictions considered for the proposed solution.

\begin{table}[H]
    \caption{Technical restrictions specification.}
    \label{table:technical_restrictions_specification}
    \centering
    \begin{tabularx}{\linewidth} {
        |>{\hsize=0.25\hsize}X|
        >{\hsize=0.75\hsize}X| }
        \cline{1-2}
        \textbf{ID}
         & \textbf{Description}                    \\ \hline
        TR-1
         & Use OpenTSDB as a Time-Series database. \\ \hline
    \end{tabularx}
\end{table}

Table~\ref{sec:technical_restrictions} shows that we have raised only one technical restriction. This technical restriction was considered because Professor Jorge Cardoso, acting as a client for this solution demanded it. OpenTSDB is the database that they are currently using in their projects at Huawei Research Center. This restriction will ease their work to introduce changes if needed.
%However, this project does not have a concrete and formally defined client, it is good to use a technology used by the people that will use the tool to ease their work and possibly introduce changes.

\section{Architecture}
\label{sec:architecture}

In this Section, the architecture is presented based on all previous topics with resource to the defined Simon Brown’s C4 Model~\cite{simon_browns_c4_model}. This approach of defining an architecture uses four diagrams: \emph{1 - Context Diagram}, \emph{2 - Container Diagram}, \emph{3 - Component Diagram} and \emph{4 - Code Diagram}. To define the architecture for our solution, only the first three representations were considered. Every representation will be exposed with a briefly explanation of the decisions taken to draw the diagram. After presenting the representations and the corresponding explanations, we will cycle thought all \gls{qa}, in order to explain where it is reflected and the considerations taken to produce the current architecture.

\subsection{Context Diagram}
\label{subsec:context_diagram}

In this Subsection the context diagram is presented. This diagram allows us to see ``the big picture'' of the system as it represents the system as a ``big box'' and its interactions with users and other systems. Figure~\ref{fig:context_diagram} presents the context diagram for this solution.

\begin{figure}[H]
    \centering
    \includegraphics[width=1.0\textwidth]{images/context_diagram.pdf}
    \caption{Context diagram.}
    \label{fig:context_diagram}
\end{figure}

From Figure~\ref{fig:context_diagram}, we can see that our solution, named \emph{Graphy \gls{otp}}, receives interactions from users, as it need someone to control its operations, analyses data from an external target system that holds the tracing information and consequently, provides extracted metrics to an external metrics visualizer component. Users can view extracted metrics from this last component.

\subsection{Container Diagram}
\label{subsec:container_diagram}

The container diagram is presented in this Subsection. This type of diagram allows us to ``zoom-in'' in the context diagram, and get a new overview of our solution. Therefore, in this diagram we are able to see a high-level shape of the software architecture and how responsibilities are distributed across it. Figure~\ref{fig:container_diagram} presents the container diagram for this solution.

\begin{figure}[H]
    \centering
    \includegraphics[width=1.00\textwidth]{images/container_diagram.pdf}
    \caption{Container diagram.}
    \label{fig:container_diagram}
\end{figure}

Figure~\ref{fig:container_diagram} contains the main containers involved in our solution. The first one, from top to bottom, is the \emph{Access Console} and this container was considered as it is needed for the user to be able interact with the \emph{Graphy \gls{api}}. This last one controls the entire OpenTracing system, uses a communication protocol to retrieve tracing information from external target system, and two databases to store the information resulted from processing tracing data -- a \gls{gdb} and a \gls{tsdb}. The second database provides metrics to be visualized in an external metrics visualizer system.

\subsection{Component Diagram}
\label{subsec:component_diagram}

This Subsection contains the last diagram, the component diagram. This type of diagram gives a more deeper vision about the system, and therefore, it reveals the main components presented in the solution. Figure~\ref{fig:component_diagram} presents the component diagram for this solution.

\begin{figure}[]
    \centering
    \includegraphics[width=1.00\textwidth]{images/component_diagram.pdf}
    \caption{Component diagram.}
    \label{fig:component_diagram}
\end{figure}

Figure~\ref{fig:component_diagram} provides us with a lower level visualization of \emph{Graphy API} container. This container is composed by a total of eight components. At its core we have \emph{Graphy Controller}, that has the responsibility of receiving requests from the user through \emph{Access Console} to control \emph{OpenTracing Processor}, \emph{Tracing Collector} and \emph{Data Analyser} components. These three components retain greater relevance for this solution. The first one has the objective of mapping tracing data, instantiating span trees and service dependency graphs into memory. The second one, collects tracing, the information that feeds this application, from local files or from external systems, e.g., Zipkin. The third one, \emph{Data Analyser} has the job of identify outliers presented in time-series metrics extracted from tracing, allowing our solution to detect anomalous services presented in distributed systems. \emph{Graph Processor} is the component for graphs handling, thus it has the capability of performing operations over graphs, e.g., subtract one graph from another graph, extract the node degrees and count connections between nodes. The remaining components, \emph{Graphs Repository} and \emph{Metrics Repository}, are used to map graphs and time-series metrics, respectively, into their corresponding databases.

To check the architecture produced, we will now cycle between both \gls{qa} and check were they are reflected in the architecture presented for this solution, explaining the trade-off involved and what were our considerations about each one.

\gls{qa}1 and \gls{qa}2 are satisfied by the fact that the system is able to collect data from an external system. Using a communication protocol where data is exchanged thought \gls{http} and exposed \gls{api}, allows to externally request little chucks of data from target systems without interfering with their normal function.

%\gls{qa}3 is satisfied when we decide to use NetworkX as the technology to process our graphs. This technology does not scale horizontally, however it has a very decent performance and it's able to retrieve a certain measure of a graph with about 100.000 nodes, in near 15 seconds\cite{networkx_speed}. From 1.000.000 spans, in normal conditions, we will never be able to get a graph of this kind of size, as with our experiments, with 100.000 spans we were able to get a graph of almost 20 nodes, and with 200.000 spans we were able to get a graph of almost 30 nodes. In the end, we are considering this time and span quantity to be sure that our tool will give us good times and ease our work of performing the research and implementation of this kind of tool.

%\gls{qa}4 is satisfied because we decided to use two databases that are scalable horizontally by design, the ArangoDb for a \gls{gdb} and the OpenTSDB for a \gls{tsdb}, both presented in the \ref{subsec:graph_database_tools} and \ref{subsec:time_series_database_tools} subsections respectively. In the end, this \gls{qa} can not be fully satisfied because we can not scale our system entirely, due to the fact of the technology that we chose to perform the graph processing. However, we have chosen this technology because it can perform much more graph algorithms, as we can see in the figure \ref{fig:graph_manipulation_and_performance_tools_diagram_comparison}, and this is much more relevant for our main purpose.

%\gls{qa}5 is satisfied by the existence of the component \textit{Logging Component}, that allows the system to perform logging of relevant information. For the technology here, we decided to use click-log\cite{click_log_doc}, a python library used for logging purposes as it has all the main capabilities needed here to perform the logging.

%\gls{qa}6, like the previous one is satisfied by the existence of a certain component, the \textit{Testing Component}, which implements all the capabilities and functionalities to perform tests and check if the systems is working correctly.

Finally, for the only technical restriction raised, we can see that it is satisfied by the usage of OpenTSDB as the main \gls{tsdb} for our solution.

With the presentation of these four sections, we conclude that our solution satisfies all the architectural drivers: quality attributes, business constraints and technical restrictions, and therefore, we may claim that the proposed architecture fits our needs as a solution.

Next Chapter,~\ref{chap:implementation_process}~-~\nameref{chap:implementation_process}, covers the implementation of the solution presented in the current chapter. All implemented algorithms and technical decisions are discussed and explained with detail.

\checkoddpage
\ifthenelse{\boolean{oddpage}}
{ % Odd page
    \newpage
    \blankpage}
{ % Even page
}
\glsresetall
\chapter{Implementation Process}
\label{chap:implementation_process}

This Chapter presents the implementation process of the proposed solution explained in previous Chapter.
%Every steps are explained considering the most important points for each implemented method. 
Three main sections are covered in this Chapter: Firstly, in Section~\ref{sec:huawei_tracing_data_set}~-~\nameref{sec:huawei_tracing_data_set}, the data provided to perform this research is presented and analysed. Secondly, in Section~\ref{sec:open_tracing_processor_component}~-~\nameref{sec:open_tracing_processor_component}, the implementation of (\gls{otp}), our proposed solution to collect and store metrics from tracing data is explained in detail with intermediate results. Finally, in Section~\ref{sec:data_analysis_component}~-~\nameref{sec:data_analysis_component}, the approach and methods for analysis of the stored observations are presented.

\section{Huawei Tracing Data Set}
\label{sec:huawei_tracing_data_set}

The starting point for this solution and every method developed within it was a data set provided by Huawei, represented by professor Jorge Cardoso. To gain access to this information, a NDA: Non-disclosure agreement was signed by both parts. This data set contains the results of tracing data gathered from an experimental \emph{OpenStack} cluster used by the company for testing purposes, and covers two days of operation. Consequently, two files were provided, one for each day. These files were generated in 10 of July, 2018 and, for protection, some fields of the data set were obfuscated during the generation process. Table~\ref{table:data_set_provided_for_this_research} contains some details about the provided data set.

\begin{table}[H]
    \caption{Huawei tracing data set provided for this research.}
    \label{table:data_set_provided_for_this_research}
    \centering
    \begin{tabularx}{\linewidth} {
        |>{\hsize=0.70\hsize}X|
        >{\hsize=1.15\hsize}X|
        >{\hsize=1.15\hsize}X| }
        \hline
        \textbf{File Date}
         & 2018-06-28
         & 2018-06-29 \\ \hline \hline
        % \textbf{File}
        % & traces-2018-06-28.jsonl
        % & traces-2018-06-29.jsonl \\ \hline
        % \textbf{Size}
        % & 130 megabytes
        % & 164 megabytes \\ \hline
        \textbf{Spans count}
         & 190 202
         & 239 693    \\ \hline
        \textbf{Traces count}
         & 64 394
         & 74 331     \\ \hline
    \end{tabularx}
\end{table}

From Table~\ref{table:data_set_provided_for_this_research}, we can see some detail regarding spans and trace counting for each day. Both files were written in JSONL format~\cite{jsonl}. This file format is an extension to the lightweight data-interchange standard \gls{json}: JavaScript Object Notation, however, in JSONL format multiple \gls{json} are separated by a new line character. Each span is presented by a single \gls{json}, therefore, each line contains a span encoded in \gls{json} format. To count spans a line count in each file was enough. To count traces, spans must be mapped to span trees, and then the total of trees represent the trace count.  Algorithms to perform this conversion are presented further, in Section~\ref{sec:open_tracing_processor_component}~-~\nameref{sec:open_tracing_processor_component}.

Span data format is defined in an open source specification called OpenTracing~\cite{open_tracing_specification}, however, companies and software developers are not obliged to follow it, thus they can produce their own span data format, leading to difficulties developing a general purpose tool for tracing analysis. Therefore, to ease the interpretation of spans presented in the data set, a file with instructions about the specification was provided. In this file, a definition was given about possible fields and their corresponding data types. A sample of the fields and their descriptions are exposed in Table~\ref{table:data_set_span_structure_definition}.

\begin{table}[H]
    \caption{Span structure definition.}
    \label{table:data_set_span_structure_definition}
    \centering
    \begin{tabularx}{\linewidth} {
        |>{\hsize=0.60\hsize}X|
        >{\hsize=1.40\hsize}X|}
        \hline
        \textbf{Field}
         & \textbf{Description}                                                                                                                                                                                                             \\ \hline \hline
        traceId
         & Unique id of a trace (128-bit string).                                                                                                                                                                                           \\ \hline
        name
         & Human-readable title of the instrumented function.                                                                                                                                                                               \\ \hline
        timestamp
         & UNIX epoch in milliseconds.                                                                                                                                                                                                      \\ \hline
        id
         & Unique id of the span (64-bit string).                                                                                                                                                                                           \\ \hline
        parentId
         & Reference to id of parent span.                                                                                                                                                                                                  \\ \hline
        duration
         & Span duration in microseconds.                                                                                                                                                                                                   \\ \hline
        binaryAnnotations
         &
        protocol - ``\gls{http}'' or ``function'' for \gls{rpc} calls; \newline
        http.url - \gls{http} endpoint; \newline
        http.status\textunderscore code - Result of the \gls{http} operation.                                                                                                                                                               \\ \hline
        annotations
         & value - Describes the position in trace (based on \emph{Zipkin} format). Could be one of the following values or other: ``cs'' (client send), ``cr'' (client receive), ``ss'' (server send) or ``sr'' (server receive); \newline
        timestamp - UNIX epoch in microseconds; \newline
        endpoint - Which endpoint generated a trace event.                                                                                                                                                                                  \\ \hline
    \end{tabularx}
\end{table}

Also, the file contained two notes. To point each one, has they are very important, we present them bellow.

\begin{enumerate}[topsep=1pt, partopsep=1pt, itemsep=5pt, parsep=5pt]
    \item Time units are not consistent, some fields are in milliseconds and some are in microseconds.
    \item Trace spans may contain more or less fields, except those mentioned here.
\end{enumerate}

From Table~\ref{table:data_set_span_structure_definition}, we get a notion about the fields that can be found in spans. These fields are defined by OpenTracing specification, therefore, is important that companies follows the specification, even if open source.

In this data set, spans are composed by: ``traceId'', ``name'', ``timestamp'', ``id'', ``parentId'' and ``duration''. These are the main required fields, because they represent the foundations for tracing data, containing the identification, relation and temporal track of the span. Also, these fields are fixed, meaning that they are always represented by the defined field name. The same can not be said from the remaining fields: ``binaryAnnotations'' and ``annotations''. These tow fields are always identified by these field names, however, their values are maps and therefore, have values stored in key - value pairs. This brings some consistency problems and we might not know clearly what is available in a span, when working with it. As said in the second point presented in the list above: ``Trace spans may contain more or less fields, except those mentioned here'', and for this reason, there is a tremendous explosion in possibilities, because there might be keys with corresponding values for some particular spans and it gets hard to generalise this in a uniform span structure.

The notion of span data only depends on the quality of communication and documentation of the ones that produce tracing. To be certain that one crafts good tracing data, there must be an implemented standard for everyone to follow. The formalization and unification of one tracing specification should be a thing to consider, for the reason that it is an endeavour to analyse inconstant fields. For example, in the data provided spans can be of two types: \gls{http} span or \gls{rpc} span, and the only field that distinguishes them is a field named ``exec'', which stands for the execution process id, and is not presented in the \gls{http} span type. Another example, fields having the same key should have one and only one measurement unit, because distributed tracing tools (like the ones presented in Subsection~\ref{subsec:distributed_tracing_tools}) are not expecting different measurement units for the same field and therefore, assume wrong values when spans have timestamps declared in milliseconds and others in microseconds, like in this case. To fix this, we decided to convert all time measurements to milliseconds.

To provide notion of how traces and spans are spread throughout time, we have used our tool, Graphy \gls{otp}, to generate two charts that represents the counting of traces and spans for each hour in each day. We decided to generate two split charts due to the simple fact that we have one file for each day. To count the number of spans in time, in this case by hour, the tool only needed to group every span by hour and count them, however for traces, the tool has more work because it needs to merge all spans in their corresponding span tree (explained in Section~\ref{sec:open_tracing_processor_component}). After having all span trees it just needs to count them, and the result is the number of traces. Note that if a span or trace starts at a given time $t1$ contained in a time-frame, and with its duration $d1$ surpassing the next time-frame $tf1$, ($t1 + d1 > tf1$), it is considered to be in the first time-frame, or by other words, only the starting time of the trace or span is considered for the counting. Figures~\ref{fig:trace_file_count_2018_06_28} and~\ref{fig:trace_file_count_2018_06_29} presents the data set traces and spans counting throughout time.

\begin{figure}[H]
    \centering
    \includegraphics[width=0.88\textwidth]{images/trace_file_count_2018_06_28_chart.png}
    \caption{Trace file count for 2018-06-28.}
    \label{fig:trace_file_count_2018_06_28}
\end{figure}

\begin{figure}[H]
    \centering
    \includegraphics[width=0.88\textwidth]{images/trace_file_count_2018_06_29_chart.png}
    \caption{Trace file count for 2018-06-29.}
    \label{fig:trace_file_count_2018_06_29}
\end{figure}

Figure~\ref{fig:trace_file_count_2018_06_28} presents the counting of traces and spans for the \nth{28} of June, 2018. In this Figure we can spot a ``pit'' in quantity from 2AM to 10AM. No explanation for this was given, however, at this point we assumed that extracting metrics from data in this time interval would produce less points, thus less resolution. This is visible in metrics presented in Figure~\ref{fig:service_calls_samples}, reproduced using \emph{Grafana}. The quantity of data for the rest of the day is somehow inconstant, however, there is no lack of data like in the previous day.

Figure~\ref{fig:trace_file_count_2018_06_29} presents the counting of traces and spans for the \nth{28} of June, 2018. In this Figure there are no ``pits'', and consequently, the quantity of information is more constant throughout time.

To summarise, this system produces an average of 5000 traces an hour and 15000 spans an hour. Also, the quantity of information provided in the second day (\nth{29} of June) is more constant, and therefore, better for analysis, than in the first day (\nth{28} of June). Nevertheless, this data set has sufficient information to study tracing data and develop methods for tracing data, and then, it is a suitable data set for this research project.

Next Chapter, \ref{sec:open_tracing_processor_component}~-~\nameref{sec:open_tracing_processor_component}, covers the explanation and algorithms used over this data set for tracing metrics extraction and quality analysis. Also, some visualizations of metrics extracted from tracing data are provided.

\newpage

\section{OpenTracing Processor Component}
\label{sec:open_tracing_processor_component}

In this Chapter, the implementation for the first component of the proposed solution, \gls{otp}, is presented and explained, hence, functional requirements defined in Table~\ref{sec:functional_requirements}, service dependency graph handling, span trees generation and methods for metrics extraction and storing from tracing data will be covered.

Starting by the first two functional requirements (FR-1 to FR-2). These require communication with distributed tracing tools, to obtain tracing data and to retrieve service dependency graphs. We have decided to use \emph{Zipkin}, as a distributed tracing tool for holding our data set, instead of Jaeger only due to simplicity in setup configuration. To setup this tool a \emph{Docker} container was instantiated in an external server. Communication methods are implemented in \emph{Tracing Collector} component. To feed information to our solution, one can use two ways: collect tracing data from local files, or export them to \emph{Zipkin} and ingest it through \gls{http} requests. This configurations can be changed by editing a configuration file provided with the solution. The configurations to edit are file locations in local machine and \emph{Zipkin} IP (Internet Protocol) address.

After collecting information from one of the two defined sources by \emph{Tracing Collector}, data is passed to \emph{Tracing Processor} which ingests and maps all the information into in memory data structures. Data can be either trace data or service dependency graphs. If it is a graph, it is transferred to \emph{Graph Processor} for process, graph metrics extraction and later storage, otherwise, it is processed in \emph{Tracing Processor} to extract defined metrics from tracing. The algorithm for metrics extraction from tracing and service dependency graphs is presented at a high abstraction level in Algorithm~\ref{alg:metrics_extraction_from_tracing}.

\begin{algorithm}
    \KwData{Trace files/Trace data.}
    \KwResult{Trace metrics written in the time-series database.}
    Connect to Time-Series database\;
    Read time\_resolution, start\_time and end\_time from configuration\;
    Read traces from trace files/trace data\;
    Post traces to Zipkin\;
    Get services from Zipkin\;
    Calculate time\_intervals using start\_time, end\_time and time\_resolution\;
    \While{time\_interval in time\_intervals}{
        Get service\_dependencies from Zipkin\;
        Build service\_dependency\_graph using service\_dependencies\;
        Extract graph\_metrics from service\_dependency\_graph\;
        \While{service in services}{
            Get traces from Zipkin\;
            Map traces in SpanTrees\;
            Extract service\_metrics from SpanTrees\;
        }
        Post graph\_metrics to Time-Series database\;
        Post service\_metrics to Time-Series database\;
    }
    \caption{Algorithm for metrics extraction from tracing.}
    \label{alg:metrics_extraction_from_tracing}
\end{algorithm}

Algorithm~\ref{alg:metrics_extraction_from_tracing} contains some core functionalities implemented in components presented in \gls{otp} solution. This algorithm aims for metrics extraction from tracing data and perform this procedure using two main data structures: service dependency graphs and tracing data mapped into SpanTrees.

\newpage

Service dependency graphs are obtained from \emph{Zipkin} and parsed directly into a \emph{NetworkX} graph structure, presented in component \emph{Graph Processor}. We decided to chose \emph{NetworkX}, a framework for graph processing written in \emph{Python}, due to tooling versatility has it contains a large implementation set of the majority graph algorithms. At this point we preferred this trade-off over processing power and scalability. \emph{Zipkin} provides service dependency graphs through an explicit endpoint -- \emph{/dependencies}, and a start and end timestamps in epoch milliseconds must be passed as parameters. The information comes in \gls{json} format as presented in Listing~\ref{lst:dependencies_zipkin_json}.

\begin{lstlisting}[caption={Zipkin dependencies result schema.},captionpos=b, label={lst:dependencies_zipkin_json}]
    [
        {
            "parent": "string",
            "child": "string",
            "callCount": 0,
            "errorCount": 0
        },
        { /* ... */ }
    ]
\end{lstlisting}

Listing~\ref{lst:dependencies_zipkin_json} shows that dependencies come in an array of \gls{json} objects. Each object contains the information about one relationship between services: parent ``from'', child ``to'' and the number of calls. Therefore, having this information grant the creation of service dependency graph using \emph{NetworkX}. Note that this information assembles a graph containing the information of system services at a specific time interval defined by provided parameters to \emph{Zipkin} \emph{/dependencies} endpoint. After having this information mapped into \emph{NetworkX} graphs in memory, their visual representation are identical to the one demonstrated in Figure~\ref{fig:service_dependency_graph}, presented in Subsection~\ref{subsec:graphs}.

SpanTrees are a representation of a trace in a tree format. Method for their creation from a span list is presented in Algorithm~\ref{alg:spans_to_span_tree}.

\begin{algorithm}
    \KwData{Span list.}
    \KwResult{Spans mapped into SpanTrees.}
    Index spans by ids from span list into SpanIndex\;
    \While{span in span list}{
        Read parentId from span\;
        Index span using parentId into SpanIndex\;
    }
    \caption{Algorithm for SpanTree mapping from spans.}
    \label{alg:spans_to_span_tree}
\end{algorithm}

Algorithm~\ref{alg:spans_to_span_tree} shows that to transform a list of spans (unordered traces) into \emph{SpanTrees}, one must index them by span id an then read every span, indexing the span using their \emph{parentId}. After applying this method, spans will be properly indexed and a list of \emph{SpanTrees} are produced. Also, a SpanTree is a representation of a trace and these structures ease tracing handling due to distinct causal relationships between spans. For example, one can use span trees to map TraceInfos. This data structure was created to hold relevant information from span trees: for example request work-flows. The process involves pinpointing requests between services, presented in spans throughout their causal relationship, and then store request paths through services, generating the corresponding request work-flow. For each span tree, one work-flow is generated, however, from root to leafs, multiple paths are possible. Note that not always do spans contain information to produce the path, and therefore, some request paths are dubious, depending only on the completeness quality of tracing. The method to produce request work-flows is described in Algorithm~\ref{alg:work_flow_type_algorithm}.

\begin{algorithm}
    \KwData{Trace files/Trace data.}
    \KwResult{\gls{csv} with unique work-flow types, their corresponding count and times.}
    Read start\_time and end\_time from configuration\;
    Read SpanList from trace files/trace data within defined time\_frame\;
    \While{have Spans in SpanList}{
        Read Span\;
        Map Span to SpanTrees\;
    }
    \While{have SpanTree in SpanTrees}{
        Read SpanTree\;
        Map SpanTree to TraceInfos\;
        Read TraceInfo\;
        Read work-flows, work-flow count, times, (others) from TraceInfo\;
        Write fields to \gls{csv} files\;
    }
    \caption{Work-flow type algorithm.}
    \label{alg:work_flow_type_algorithm}
\end{algorithm}

The method described in Algorithm~\ref{alg:work_flow_type_algorithm} aims to use tracing to produce span trees, and then generate TraceInfos to retrieve request work-flow paths.

These two data structures, service dependency graphs and span trees, are the foundations to extract metrics from tracing data, satisfy the functional requirements presented in Section~\ref{sec:functional_requirements} and answer the final research questions defined in Section~\ref{sec:research_questions}.

The metrics that \gls{otp} is able to extract from tracing data, for a defined time interval, are the following:

\begin{enumerate}
    \item Number of incoming/outgoing service calls;
    \item Average response time by service;
    \item Service connection, i.e., other services invoking and being invoked by the system, i.e., the service dependency graph variation.
    \item Service degree (in/out/total);
    \item Service \gls{http} status code ratio. (sum of success or failure count over total status code count)
\end{enumerate}

These metrics are all related with time and represent observations of values extracted from tracing data, therefore, as time-series metrics they are stored in a \gls{tsdb}. Explanation of used technology and procedure is provided later on this Section.

Table~\ref{table:research_questions_frs_and_metrics} relates each metric with a functional requirement, and correspondent final research question. Functional requirements are identified by an \emph{id} from Table~\ref{table:functional_requirements_specification}.

\begin{table}[H]
    \caption{Relations between final research questions, functional requirements and metrics.}
    \label{table:research_questions_frs_and_metrics}
    \centering
    \begin{tabularx}{\linewidth} {
        |>{\hsize=0.75\hsize}X|
        >{\hsize=0.60\hsize}X|
        >{\hsize=1.65\hsize}X|}
        \hline
        \textbf{Research} \newline \textbf{Question}
         & \textbf{Functional} \newline \textbf{Requirements}
         & \textbf{Metrics}                                   \\ \hline \hline
        1. Is there any anomalous service?
         & FR-5; \newline
        FR-5; \newline
        FR-6.
         & Number of incoming service calls; \newline
        Number of outgoing service calls; \newline
        Average response time by service.                     \\ \hline
        \textcolor{gray}{2. What is the overall reliability of the service?}
         & FR-7; \newline
        FR-8.
         & No metric extracted; \newline
        Service \gls{http} status code ratio.                 \\ \hline
        \textcolor{gray}{3. Which service consumes more time when considering the entire set of requests?}
         & FR-9; \newline
        FR-10.
         & Service degree; \newline
        Service dependency graph variation.                   \\ \hline
        %4. How can we measure the quality of tracing?
        % & FR-11; \newline
        % FR-12.
        % & 1                                                  \\ \hline
    \end{tabularx}
\end{table}

For Table~\ref{table:research_questions_frs_and_metrics}, only functional requirements from numbers 5 to 10 were considered, due to being the ones related with metrics extraction. As said before, only question number 1 was considered for metrics extraction. The remaining, defined at grey colour, were implemented and \gls{otp} extracts them, however, they were not further analysed in this research. Almost all functional requirements have one metric associated except one, \emph{FR-7}. This functional requirement was implemented, and our solution allows to generate work-flow paths from tracing data, however, no metric was defined. Nevertheless, the implementation of this functionality helped us understanding results for the first final research question -- Method for work-flow generation from tracing data is explained Algorithm~\ref{alg:work_flow_type_algorithm}.

Span trees are a representation of causal relationship between spans. Two types of time based metrics are extracted from span trees: 1. Average response time by service in time; and 2. Service \gls{http} status code ratio in time. To extract the first metric type, \emph{duration} and \emph{annotations/endpoint/serviceName} values presented in spans , when defined, are used to calculate the average response time by service. For each span tree a list of services and their corresponding average times are obtained. After gathering all values from every span tree presented in the defined time-frame, the values are merged and posted to the \gls{tsdb}. The second metric, is extracted through a calculation of status codes ratio by each service. For this, \emph{binaryAnnotations/http.status\_code} and \emph{annotations/endpoint/serviceName} values are used. Also, equally to the previous metric, values are merged and posted to the \gls{tsdb}.

Service dependency graphs are a representation of dependencies of services at a specific time-frame. Three types of time based metrics are extracted from service dependency graphs: 1. Number of incoming/outgoing service calls in time; 2. Entry/exit of services in time (service dependency graph node variation); and 3. Service degree (in/out/total) in time. To extract the first metric type, the values in between (Edges) services (Nodes) are retrieved. These values are dispatched for storage with service name, flow indication (incoming/outgoing), timestamp and number of calls. The second metric type is extracted having two successive graphs and performing their difference. For example, if $Graph A$ has nodes ${A, B, C}$ and $Graph B$, nodes ${A, C, D, E}$, the difference between them will result in two service entries ${D, E}$ and one exit. Last metric type, service degree, is extracted by retrieving the number of connections from each service. For example, consider that $Graph C$ has a service $A$ connected from itself to services ${B, C, D}$. In this graph, service $A$ has an out degree of three and an in degree of zero. The remaining services have an out degree of zero and an in degree of one. Methods to extract these metrics are implemented in \emph{Graph Processor} and resource to \emph{NetworkX} to handle graph structures. All these metrics are then posted to the \gls{tsdb}.

At this point, our solution is able to retrieve and store time-series metrics from tracing data. For the \gls{tsdb}, we have decided to use \emph{OpenTSDB}, due to technical restrictions imposed in Section~\ref{sec:technical_restrictions}. There was a client implementation for usage in \emph{Python}, however, the support was not good due to lack of updates and clear documentation. For this reason, we decided to implement our own \emph{OpenTSDB} client in \emph{Python} using their \gls{api} specification, -- \emph{Metrics Repository} component. Later, when all implementation from tracing collection through trace metrics storage in the \gls{tsdb}, we used a browser metrics visualizer. To do this, a \emph{Docker} container with \emph{Grafana}, a data visualization tool capable of rendering time-series metrics in charts and present them in dashboards. The decision to use this tool, was due to easy to setup and integrated compatibility with our \gls{tsdb}. We just needed to create a container and, through a url configuration in \emph{Grafana}, we established a link to the \gls{tsdb}.

Figures~\ref{fig:service_calls_samples},~\ref{fig:service_dependency_variation},~\ref{fig:service_avg_response_time_samples}, and~\ref{fig:service_status_code_ratio_samples}, contain sample representations of extracted time-series metrics stored in our \gls{tsdb}.

\begin{figure}[H]
    \centering
    %\includegraphics[width=1.00\textwidth]{images/service_calls.pdf}
    \includegraphics[width=15cm, height=9cm]{images/service_calls.pdf}
    \caption{Service calls samples.}
    \label{fig:service_calls_samples}
\end{figure}

\begin{figure}[H]
    \centering
    \includegraphics[width=1.00\textwidth]{images/service_dependency_variation.pdf}
    \caption{Service dependency variation samples.}
    \label{fig:service_dependency_variation}
\end{figure}

\begin{figure}[H]
    \centering
    \includegraphics[width=1.00\textwidth]{images/service_avg_response_time.pdf}
    \caption{Service average response time samples.}
    \label{fig:service_avg_response_time_samples}
\end{figure}

%\begin{figure}[H]
%    \centering
%    \includegraphics[width=1.00\textwidth]{images/service_degrees.pdf}
%    \caption{Service degree samples.}
%    \label{fig:service_degree_samples}
%\end{figure}

\begin{figure}[H]
    \centering
    \includegraphics[width=1.00\textwidth]{images/service_status_code_ratio.pdf}
    \caption{Service status code ratio samples.}
    \label{fig:service_status_code_ratio_samples}
\end{figure}

Figure~\ref{fig:service_calls_samples} represent samples about the number of service request calls metric. In this Figure, we have 9 plots, three in each row, representing three variations (incoming, outgoing and total) of this metric for one service. In this metric we can clearly see the lack of information presented in tracing for the beginning of the first day.

Figure~\ref{fig:service_dependency_variation} contain samples about service dependency variation, one for each metric (gain, loss and total). Total are the result of $gain - loss$. Gain stands for the number of new service entries in system, and loss, represent the number of service exits in system.

Last Figure,~\ref{fig:service_status_code_ratio_samples}, shows the gathering of status code ratio samples for three distinct services. The ratio varies from 0.0 to 1.0, and represent the proportion of status code groups (2xx -- Success, 4xx -- Client error and 5xx -- Other errors).

Also, service dependency graphs are stored for further access after being processed by \emph{Graph Processor} component. We have decided to use \emph{ArangoDB} as our \gls{gdb}. This decision was based in the ``Multi data-type support'' provided by this database, allowing us to extend our graph structures to whatever we wanted, enhancing our graph storage possibilities and relieving the implementation from parsing data-types. This database has a \emph{Python} client, \emph{pyArango}~\cite{pyarango}, which revealed lack of features, leading to propositions for functionality creation and issue declarations in GitHub. However, the answers were not pleasant due to lack of support and people to maintain the project~\cite{arango_issues}. This have lead to some difficulties when implementing \emph{Graphs Repository} component in \gls{otp}. Difficulties from storing graphs with custom names to custom graph retrieval were felted. The solution was to fork the project, perform changes and use our custom \emph{pyArango} client. This changes were committed for review to the original project. We could not mitigate these problems in advance because they were only perceived when using the client.

%Also, service dependency graphs are processed and stored in a \gls{gdb}, however, they do not have further use for this research. The decision to keep them and store them, eases their retrieval and further usage on upcoming projects.

After presenting the first component, \gls{otp}, from our proposed solution, next Section~\ref{sec:data_analysis_component}~-~\nameref{sec:data_analysis_component} covers the implementation of the second component presented in our solution.

\section{Data Analysis Component}
\label{sec:data_analysis_component}

In this Section, the implementation of the second component presented in our proposed solution, ``Data Analysis'' component, is presented and expected outcomes from each analysis are discussed.

``Data Analysis'' component has the main objective of detecting anomalies, presented in services, using time-series metrics extracted from tracing data using the component presented in previous Section and perform tracing quality analysis.

In our implementation, this component is detached from the remaining components, however, in architectural terms we have decided to place it has being part of the overall solution. This is because there is nothing preventing total integration with other components presented in the solution. The reason to implement these methods detached from the remaining, was to ease our research path and increase flexibility. This means that, to ease our data exploration, implement these methods in Notebooks detached from the overall components, allowed us to change code effortlessly and conceded focus on methods development. Jupyter Notebook~\cite{jupyter_notebooks} was the notebook chosen for method implementation, hence, one server was created to hold our implementations in notebooks.

In this case, extracted time-series data belong to unlabelled data group. Data can belong to unlabelled or labelled groups. Unlabelled data are information sampled from of natural, or human-created artefacts, that one can obtain from observing and record values. In this group, there is no ``explanation'' for each piece of data, as it just contains the data, and nothing else. Labelled data typically takes a set of unlabelled data and augments each piece of data with some sort of meaningful ``tag'', ``label'', or ``class'' that is somehow informative or desirable to know. For example, for this solution, having labelled data would help in identifying anomalies presented in our data set. However, only unlabelled data was provided, and therefore, we needed to work with unlabelled data and perform anomaly detection with it~\cite{Kothari}.

So, the approach was to use processed data produced from \gls{otp}, and perform the analysis using ``Data Analysis'' component to point out service problems and perform tracing quality analysis, as defined in Figure~\ref{fig:proposed_approach}, to answer questions defined in Chapter~\ref{chap:research_objectives_and_approach}:

\begin{enumerate}
    \item Is there any anomalous behaviour in the system? (If yes, where?);
    \item How can we measure the quality of tracing?
\end{enumerate}

To answer the first question, using our proposed solution, one must use metrics extracted from tracing data, namely the number of incoming / outgoing requests and the average response time for each service. These metrics are time-series metrics, and therefore, anomaly detection using unsupervised learning algorithms are the way to do it~\cite{Brillinger2006}. Metrics were obtained using methods defined in our \emph{OpenTSDB} client, implemented in \emph{Metrics Repository} component.

After extracting these metrics, they are allocated in a data structure called \emph{dataframe} from \emph{Pandas}, an open source library that provides high-performance, easy-to-use data structures and data analysis tools. This library was chosen due to being one of the most used and popular for this purpose~\cite{pandas} -- data science and data analysis. A \emph{Dataframe} is a two-dimensional size-mutable, potentially heterogeneous tabular data structure with labelled axes (rows and columns).  In these data structures, values from time-series metrics were stored in columns: \emph{timestamp} (index), \emph{datetime}, \emph{number\_of\_incoming\_requests}, \emph{number\_of\_outgoing\_requests} and \emph{average\_response\_time}. In the end, a list of \emph{dataframes} are created, one \emph{dataframe} for each service.

Before performing an analysis to detect if there are outliers presented in our data, the information must be checked and tested to verify if data have missing values. This is done because metrics are extracted from multiple sources and thus generates missing values. For example, we may have missing information for one of the three features in a row of one \emph{dataframe}. Missing values are a pain in data analysis and are represented by \emph{NaN} in \emph{dataframes}, and for this reason,one can not apply anomaly detection algorithms over data with missing values. To fulfil missing information there are two approaches:

\begin{enumerate}
    \item Remove rows with missing values, which degrades the overall data and may result in insufficient data;
    \item Impute missing values, however, it may be dangerous because it introduces "wrong" observations.
\end{enumerate}

We decided to impute missing values because there were too keep information quantity. However, there are multiple ways for imputation of missing values into time-series data, depending on factors of trend and seasonality. Trending is the increasing or decreasing value in the series, and seasonality is the repeating short-term cycle in the series~\cite{Brillinger2006}. Figure~\ref{fig:methods_to_fulfil_time_series_data} shows the path to chose the correct method to fulfil information in time-series data.

\begin{figure}
    \centering\includegraphics[width=0.8\linewidth]{images/methods_to_fulfil_time_series_data.pdf}
    \caption{Methods to handle missing data~\cite{Swalin2019}.}
    \label{fig:methods_to_fulfil_time_series_data}
\end{figure}

So, before applying the method, our component tests the data to chose the correct method to fulfil data. Figure~\ref{fig:trend_seasonality_results} contains trend and seasonality sample tests performed over our data.

\begin{figure}
    \centering\includegraphics[width=0.8\linewidth]{images/trend_seasonality_results.pdf}
    \caption{Trend and seasonality results.}
    \label{fig:trend_seasonality_results}
\end{figure}

Figure~\ref{fig:trend_seasonality_results} shows that there are clearly trends in our data, however, no seasonality was detected. For this reason, the selected method to fulfil data presented in \emph{dataframes} is Linear Interpolation -- Figure~\ref{fig:methods_to_fulfil_time_series_data}.

These \emph{dataframes} are then processed by an unsupervised learning algorithm to detect if there are outliers. For the unsupervised learning algorithm there were: Isolation Forests and OneClassSVM~\cite{Zhou2017}. The first one uses binary decision trees to isolate data points and identify outliers presented in the data set, the second one, generates density areas using max-margin methods, i.e. they do not model a probability distribution, hence the idea is to find a function that is positive for regions with high density of points, and negative for small densities, identifying outliers presented in data. Figure~\ref{fig:isolation_forests_and_oneclasssvm_comparison} displays the error comparison of these two methods.

\newpage

\begin{figure}[H]
    \centering\includegraphics[width=0.8\linewidth]{images/isolation_forests_and_oneclasssvm_comparison.pdf}
    \caption{Isolation Forests and OneClassSVM methods comparison~\cite{isolation_forests_and_oneclasssvm_comparison}.}
    \label{fig:isolation_forests_and_oneclasssvm_comparison}
\end{figure}

From Figure~\ref{fig:isolation_forests_and_oneclasssvm_comparison}, isolation forests prove to be a better method for outlier detection because, from this test, it resulted in fewer errors as it did not construct a parametric representation of the search space. For this reason, we decided to use Isolation Forests, to detect and identify outliers presented in time-series metrics extracted from tracing data. To implement Isolation Forests method we used Scikit-Learn, a library full of simple and efficient tools for data mining, data analysis and machine learning. All configurations used from this library to implement Isolation Forests were setted to default. Therefore, Algorithm~\ref{alg:anomaly_detection} presents the whole process to identify anomalous services presented in the system.

\begin{algorithm}
    \KwData{Processed data from tracing using \gls{otp}.}
    \KwResult{Report, in \gls{csv} file, containing identified anomalous services and correspondent times.}
    Read start\_timestamp, end\_timestamp, db\_settings from configuration\;
    Connect to \gls{tsdb}\;
    Retrieve metrics from \gls{tsdb} using database connection, start\_timestamp and end\_timestamp\;
    Create dataframes with metrics data\;
    Perform data imputation over dataframes\;
    Feed Isolation Forests with metric columns from dataframes\;
    Fire Isolation Forests method (Adds new column ``anomaly'' with -1 ``Anomalous'' or 1 ``Non-anomalous'')\;
    Filter ``anomaly'' column with -1 values from dataframes into anomalous\_dataframes\;
    Write report with anomalous service names and times from anomalous\_dataframes data\;
    \caption{Anomalous service detection algorithm.}
    \label{alg:anomaly_detection}
\end{algorithm}

Algorithm~\ref{alg:anomaly_detection} contains all the process explained above. The final outcome from this algorithm is a report containing all anomalous services and correspondent times identified. Also, later we decided to study further the pattern observed in anomalous regions. For this, the approach was to use the algorithm defined in ~\ref{alg:work_flow_type_algorithm} to analyse what happens to work-flows in ``anomalous'' and ``non-anomalous'' regions.

To answer the second question, it requires to perform a structural and time coverage analysis. For the first analysis, the approach is to define a specification schema based on \emph{OpenTracing} open source specification. This schema aims to test span structures in order to detect structural problems present in spans, e.g., missed fields, wrong data types, typos presented in structure. The method implemented is presented in Algorithm~\ref{alg:span_structure_analysis_algorithm}.

\begin{algorithm}
    \KwData{Trace files/Trace data.}
    \KwResult{\gls{csv} file reporting span structure analysis.}
    Read specification from open\_tracing\_specification\_schema.json\;
    \While{not end of tracing}{
        Read Span\;
        Check Span against specification\;
    }
    Write results from ``Check'' to \gls{csv} file\;
    \caption{Span structure analysis algorithm.}
    \label{alg:span_structure_analysis_algorithm}
\end{algorithm}

As we can see in Algorithm~\ref{alg:span_structure_analysis_algorithm}, our method aims to produce a report containing the results of span structural analysis. To do this, first it needs to read the \emph{OpenTracing} specification schema. This schema is written in a \gls{json} file, where the fields are annotated with tags: \emph{required}, \emph{data-type: <string, int, other>} and others. \gls{json} Schema~\cite{json_schema_library} was the library used to verify if each span complies with the specification. For the second analysis, the approach is to use spans presented in trace data to analyse the coverage of each trace. Figure~\ref{fig:trace_time_coverage} presents an example for time coverage in a trace.

\begin{figure}[H]
    \centering\includegraphics[width=0.8\linewidth]{images/trace_time_coverage_example.pdf}
    \caption{Trace time coverage example.}
    \label{fig:trace_time_coverage}
\end{figure}

Figure~\ref{fig:trace_time_coverage} gives us an example in which we have a trace with a root span of 100 milliseconds of duration, and this root span has two children spans, one with $50ms$, the other one with $10ms$, the entire trace has a coverage of $(50+10)/100=60\%$. This method is applied to every trace, and the results are the stored in a \gls{csv} file to be plotted for visualisation. In this case we apply it and split the results by service, with the objective of perceive the time coverability of tracing in each service. The method is presented in Algorithm~\ref{alg:trace_coverability_analysis}.

\begin{algorithm}
    \KwData{Trace files/Trace data.}
    \KwResult{\gls{csv} file for each service reporting the coverability analysis.}
    Read start\_time and end\_time from configuration\;
    Get services from Zipkin\;
    \While{service in services}{
        Get traces from Zipkin using service, start\_time and end\_time\;
        Map traces in SpanTrees\;
        Calculate trace\_coverability using SpanTrees\;
        Write trace\_coverability to \gls{csv} file\;
    }
    \caption{Trace coverability analysis algorithm.}
    \label{alg:trace_coverability_analysis}
\end{algorithm}

Algorithm~\ref{alg:trace_coverability_analysis} uses \emph{SpanTrees} to calculate \emph{trace\_coverability}, this is due to causal relationships presented in these trees. As explained above, through Figure~\ref{fig:trace_time_coverage}, one must have a trace mounted in span relationships (span trees), to know when a span is child of another, and be able to calculate the coverability presented in a trace. This method performs this calculation for every service and, in the end, stores information about trace coverability into a \gls{csv} file. This file is later used to produce plots about the service trace coverability. What is expected from this method is that we achieve a plotting, where every service has a counting of traces that cover a certain amount of time.

%\todo{TALK A LITTLE BIT MORE???}

To summarise, this tools gathers processed data and time-series data from our \gls{tsdb}, extracted using \gls{otp} from original trace information. Then it perform data imputation to solve missing values problems, analyses resulting data using Isolation Forests, an unsupervised multiple feature machine learning algorithm, to identify outliers presented in our extracted metrics, and therefore, detect anomalies presented in services, identifying their occurrences in time. Also, this tools uses tracing to perform an analysis about the structure of spans presented in tracing, and uses processed data from \gls{otp}, to perform an analysis of time coverage provided by tracing data.

Next Chapter~\ref{chap:results_analysis_and_limitations}, we will cover results obtained by this component, discuss these results and present \emph{OpenTracing} data limitations.

\checkoddpage
\ifthenelse{\boolean{oddpage}}
{
    \newpage
    \blankpage
}
{
    % Even page
}
\glsresetall
\chapter{Results, Analysis and Limitations}
\label{chap:results_analysis_and_limitations}

In this Chapter we present the results gathered from the ``Data Analysis'' component presented in Chapter~\ref{chap:proposed_solution}~-~\nameref{chap:proposed_solution}, to answer the questions defined in Section~\ref{sec:research_questions}. Results for both questions, ``1. Is there any anomalous service?'' and ``2. How can we measure the quality of tracing?'', are presented as well as a brief discussion regarding both results.

\section{Anomaly Detection}
\label{sec:anomaly_detection}

For the first question, the approach was use the OTP tool to extract metrics from tracing data to further analyse it using the unsupervised learning algorithm. The implemented algorithm used for metrics extraction is presented in Algorithm~\ref{alg:metrics_extraction_from_tracing}.

After extract metrics, a tool for metrics visualisation (e.g., Grafana) can be used to visualise the metrics from the Time-Series database.

\todo{Explain path until arrive at data in Figure~\ref{fig:comparison_anomalous_non_anomalous_regions} -- ADD FIGURE with anomalies.}

Figure~\ref{fig:comparison_anomalous_non_anomalous_regions} provides a representation of two time-frame samples, one for the ``anomalous'' region, and the other for the ``non-anomalous'' region considering the same service. In this samples we retrieved data to analyse and give answers to the first question. We considered three features (as shown in the samples bellow): the number of incoming requests, the number of outgoing requests and the average response time. The sample resolution for the time-frame is 10 minutes.

\begin{figure}
  \centering\includegraphics[width=1.0\linewidth]{images/result_samples_for_service_a.pdf}
  \caption{Comparison between ``Anomalous'' and ``Non-Anomalous'' service time-frame regions.}
  \label{fig:comparison_anomalous_non_anomalous_regions}
\end{figure}

As we can see in Figure~\ref{fig:comparison_anomalous_non_anomalous_regions}, there is a clear difference between anomalous and non-anomalous regions. There is a drastic change in the range of values between the anomalous and non-anomalous regions, where the maximum for each feature changes drastically and therefore, outliers are visible and evident in the observations. In the anomalous samples, it is possible to notice a clear crowding of points near the origin point of the chart and some outliers in the upper-left and down-right regions of the chart. On the other side, in the non-anomalous samples, all that is possible to notice is the crowding of points near the origin point of the chart. The crowding of points is what is expected to be the normal behaviour for services, which means that is expected that the service can handle the load with good response times. Furthermore, after this observations, what is expected is to investigate what these points represent and what is causing this unexpected increment in the number of incoming/outgoing requests and the average response time. There are two anomalous situations observed:

\begin{enumerate}
  \item Services are increasing the response time when there are few incoming/outgoing requests.
  \item Services are receiving more incoming/outgoing requests, however it is having a good response time.
\end{enumerate}

The first situation is much worse than the second one. The expectation is that services can handle more requests and keep the average response time. The worst case scenario would be to find points in the upper-right section of the charts, however this was not observed in this tracing data.

After this, and to study both situations, an analysis of trace request work-flow types was performed. The objective of this analysis is to perceive if there is some strange occurrences in request work-flow paths. To be able to do this, the OTP must be able to get the tracing data and map each unique trace work-flow for the given time-frame. The algorithm is presented in Algorithm~\ref{alg:work_flow_type_algorithm}:

\todo{Mentioned in Section 6.1}
\begin{algorithm}
  \KwData{Trace files/Trace data.}
  \KwResult{\gls{csv} with unique workflow types, their corresponding count and times.}
  Read start\_time and end\_time from configuration\;
  Read SpanList from trace files/trace data within defined time\_frame\;
  \While{have Spans in SpanList}{
    Read Span\;
    Map Span to SpanTrees\;
    Read SpanTree\;
    Map SpanTree to TraceInfos\;
    Read TraceInfo\;
    Read workflows, workflow count, times, (others) from TraceInfo\;
    Write fields to \gls{csv} files\;
  }
  \caption{Work-flow type algorithm.}
  \label{alg:work_flow_type_algorithm_2}
\end{algorithm}

As presented in algorithm~\ref{alg:work_flow_type_algorithm}, parameters from TraceInfo are written to the \gls{csv} files. These files are then processed in the ``Data Analyser'' component. Interesting results, from ``Anomalous'' and ``Non-Anomalous'' regions of the services work-flow types are presented in Figure~\ref{fig:work_flow_type_analysis}.

\begin{figure}
  \centerline{\includegraphics[width=1.0\linewidth]{images/workflow_type_count.pdf}}
  \caption{Comparison between ``Anomalous'' and ``Non-Anomalous'' service work-flow types.}
  \label{fig:work_flow_type_analysis}
\end{figure}

One interesting thing to notice and that proves this method is that, in the anomalous regions, more quantity and more types of request work-flow types were observed. The next step was to check what was causing this by retrieving the most ``called'' work-flow, however the results were not good because of the completeness of the tracing data. The flows were not relevant for a further analysis because they were just calls from point A to B. At this point, and for this question, it is possible to say that this data set was exhaustively analysed, and an improvement of the tracing data should be a path to take. One point to note for future work is to test this method with other tracing data to understand the root cause of the anomalous behaviour.

\section{Trace Quality Analysis}
\label{sec:trace_quality_analysis}

For the second question, the approach was the same as in the previous question, use the OTP to process the tracing data and gather the results to be further analysed in the ``Data Analysis'' component. However, in this case, the results obtained by the first component were directly used by the second component.

In this question the analysis is divided in two procedures as explained in Chapter~\ref{chap:proposed_solution}.

The first procedure aims to check if the spans comply with the OpenTracing specification. The method is rather simple and is presented in Algorithm~\ref{alg:span_structure_analysis_algorithm}.

\todo{Mentioned in Section 6.1}
\begin{algorithm}
  \KwData{Trace files/Trace data.}
  \KwResult{\gls{csv} file reporting span structure analysis.}
  Read specification from open\_tracing\_specification\_schema.json\;
  \While{not end of tracing}{
    Read Span\;
    Check Span against specification\;
  }
  Write results from ``Check'' to \gls{csv} file\;
  \caption{Span structure analysis algorithm.}
  \label{alg:span_structure_analysis_algorithm_2}
\end{algorithm}

The results obtained by this method were that every span structure complies with the specification. This is not a very good test because the specification of the OpenTracing is not very strict and therefore, the created method for testing does not provide a very accurate kind of results. For example, the units for timestamps are not uniform, one can use milliseconds and in other field of a span presented in the same trace, other can use microseconds. This leads to problems in time measurements and is not covered by this test. The redefinition of the specification is discussed in Chapter~\ref{chap:conclusion_and_future_work}.

\todo{Check bellow because it was explained in Ch.6.}

The second procedure aims to check if tracing covers the entire time of the root spans. For a simple example, if we have a trace with a root span of 100 milliseconds of duration, and this root span has two children spans, one with $50ms$, the other one with $10ms$, the entire trace has a coverage of $(50+10)/100=60\%$. This method is applied to every trace, and the results are plotted for visualisation. In this case we apply it and split the results by service, with the objective of perceive the time coverability of tracing in each service. The method is presented in Algorithm~\ref{alg:trace_coverability_analysis} and the corresponding results, regarding two different services, are presented in Figure~\ref{fig:services_coverability_analysis}.

\todo{Mentioned in Section 6.1}
\begin{algorithm}
  \KwData{Trace files/Trace data.}
  \KwResult{\gls{csv} file for each service reporting the coverability analysis.}
  Read start\_time and end\_time from configuration\;
  Get services from Zipkin\;
  \While{service in services}{
    Get traces from Zipkin using service, start\_time and end\_time\;
    Map traces in SpanTrees\;
    Calculate trace\_coverability using SpanTrees\;
    Write trace\_coverability to \gls{csv} file\;
  }
  \caption{Trace coverability analysis algorithm.}
  \label{alg:trace_coverability_analysis_2}
\end{algorithm}

\begin{figure}
  \centering{\includegraphics[width=1.0\linewidth]{images/service_trace_coverability_analysis.pdf}}
  \caption{Services coverability analysis.}
  \label{fig:services_coverability_analysis}
\end{figure}

Figure~\ref{fig:services_coverability_analysis} allows the user to visualise the tracing coverability, in terms of how many the overall tracing covers their entire execution duration. The most important thing to notice is the presence of higher bar values in $60\%-100\%$ regions. This means that coverability for this tracing could be better, but in overall is good. What is expected by the result of this kind of analysis is that the coverability of tracing remains closer to the last interval $(90\%-100\%)$, which means that our service is fully/almost fully covered by this kind of data and therefore, the analysis of this data is worthy and the results provided by the usage of this data are trusty. From this data set, the results for each service where close to what is shown by this figure. From this analysis, one thing to improve is to develop a method to analyse the gathered results in order to detect traces that do not cover their duration with respect to a predefined threshold. This would allow developers to improve their tracing coverage.

\section{Limitations of OpenTracing Data}
\label{sec:limitations_of_opentracing_data}

\todo{...}

\checkoddpage
\ifthenelse{\boolean{oddpage}}
{ % Odd page
\newpage
\blankpage}
{ % Even page
}
\glsresetall
\chapter{Conclusions}
\label{chap:conclusions}

This chapter presents the conclusions to this whole work. Therefore in this chapter we decided to split it into three sections, first there is a brief reflection about this whole work in \ref{sec:brief_reflection} - \nameref{sec:brief_reflection}, in second a section about the future work that can follow the whole research carried out by this investigation.

\section{Brief Reflections}
\label{sec:brief_reflections}

This thesis presents an attempt to further the field of distributed system based tracing analysis.

// TODO: Talk a bit about everything

\todo{A reflection about the tools and methods produced and the open paths from this whole research are exposed. Also a reflection of the main difficulties felted with this research are presented}

\section{Future Work}
\label{sec:future_work}

// TODO: Talk a bit about the possible work in this field and what can be achieved using this work.

\todo{the future work that can be addressed considering this work is properly explained taking into consideration what is said in the previous section}

\section{Concluding Research Questions}
\label{sec:concluding_research_questions}

// TODO: Talk a bit about everything


%-------------------------------------------------------------------------------------------------
\checkoddpage
\ifthenelse{\boolean{oddpage}}
{ % Odd page
\newpage
\blankpage}
{ % Even page
}
%-------------------------------------------------------------------------------------------------

\printbibliography[title={References}]
\newpage

% TODO: Uncomment the following for the final report delivery
%\begin{appendices}
%\input{chapters/9-appendices}
%\end{appendices}
%------------------------------

\end{document}